\section{Separation of inequalities} % -----------------------------------------
\label{sec:las:separation}

\newcommand{\ox}     {\overline{x}}
\newcommand{\ov}     {\overline{v}}
\newcommand{\oz}     {\overline{z}}
\newcommand{\fracsol}{$(\ox_i^k, \ov_i^\ell, \oz^k)$}

\subsection{Introduction} % ....................................................
\label{ssc:las:soi:introduction}

The section will start with formulation \slla in mind, but its similarity with
\sllb will cause the conclusions for \slla to also be applicable to \sllb.  Let
us thus consider a fractional solution of \slla, \fracsol. The strategy will be
to find integers $r_i^k$ and sets $Q_i^k$ minimising the value of the right-hand
side of \eqref{eq:sll1-inequalities}. he resulting inequality will be added to
\slla iff it separates the fractional solution from the feasible set. That is,
iff it is violated by \fracsol.

Note that the sum of the right-hand side of \eqref{eq:sll1-inequalities} may be
minimised by choosing each $r_i^k, Q_i^k, \forall\is$ so as to minimise the
summand corresponding to product $i$. Therefore, for each $\is$ the following
problem must be solved
\begin{equation}
    \min_{r \in       \{0, \ldots, \sigma(k)\}\\
          Q \subseteq \{1, \ldots, r - 1\}} \left(
        b^r \ox_i^k
        + \sum_{\ell = r + 1}^{\sigma(k)}
            \left(b^\ell - b^r\right)\ov_i^\ell
        + \sum_{\ell \in Q}
            \left(b^\ell - b^r\right)\left(\ox_i^k + \ov_i^k - 1\right)
    \right).
    \label{eq:minimisation-problem}
\end{equation}

We have defined $r = r_i^k$ and $Q = Q_i^k$ in order to simplify the notation.
Note that the result of this minimisation problem also minimises the right-hand
side of \eqref{eq:sll2-inequalities} for $x = \ox_i^k$ and $v = \ov_i^\ell$.
This will permit in \sllb the same separation schema that presented for \slla.
Studying \eqref{eq:minimisation-problem}, the following conclusion is reached:

\begin{lemma}
    For a fixed $r$, a possible value of $Q^r \subseteq \{1, \ldots, r-1\}$
    minimising \eqref{eq:minimisation-problem} is
    \[
        Q^r = \{
            \ell \in \{1, \ldots, r-1\}
            ~:~
            \ox_i^k + \ov_i^\ell > 1
        \}.
    \]
\end{lemma}

\begin{proof}
    It suffices to note that the only summand in \eqref{eq:minimisation-problem}
    depending on $Q$ is
    \[
        \sum_{\ell \in Q} \left(b^\ell - b^r\right)\left(\ox_i^k + \ov_i^\ell - 1\right).
    \]
    Since $Q \subseteq \{1, \ldots, r - 1\}$, the term $b^\ell - b^r$ is always
    negative in the sum above. Therefore, an optimal $Q^r$ contains all $\ell$
    such that
    $
        \ox_i^k + \ov_i^\ell - 1 > 0
    $,
    and no $\ell$ for which
    $
        \ox_i^k + \ov_i^\ell - 1 < 0
    $.
\end{proof}

This way, defining $S(r)$ to be the objective value of
\eqref{eq:minimisation-problem} when $Q = Q^r$, we may devote ourselves to
studying the following problem, which is equivalent to
\eqref{eq:minimisation-problem}:

\begin{equation}
    \min_{r \in \{0, \ldots, \sigma(k)\}} S(r)
    \label{eq:alt-minimisation-problem}
\end{equation}

\subsection{Minimisation of $S$} % .............................................
\label{ssc:las:soi:minimisation}

In this section, $S$'s minimum will be found. The objective is to prove that $S$
has a very concrete shape: it is first decreasing, then increasing. We will
start presenting the following auxiliary result characterising the first
difference of $S$:

\begin{lemma}
    If $r < \sigma(k)$, then
    \[
        S(r + 1) - S(r) =
            \left(b^{r + 1} - b^r\right)
            \left(
                \ox_i^k
                + \sum_{\ell = r + 1}^{\sigma(k)} \ov_i^\ell
                + \sum_{\ell \in Q^{r + 1}} (1 - \ox_i^k - \ov_i^\ell).
            \right)
    \]
\end{lemma}

\begin{proof}
    The result can be obtained by a direct derivation:
    \begin{eqnarray*}
        &S(r + 1)
             &-  S(r)\\
        &    &=  \left(
                    b^{r+1}\ox_i^k
                    + \sum_{\ell = r + 2}^{\sigma(k)}
                        \left(b^\ell - b^{r + 1}\right)\ov_i^\ell
                    + \sum_{\ell \in Q^{r + 1}}
                        \left(b^\ell - b^r\right) \left(\ox_i^k + \ov_i^\ell - 1\right)
                \right)\\
        &    &\quad -\left(
                    b^r\ox_i^k
                    + \sum_{\ell = r + 1}^{\sigma(k)}
                        \left(b^\ell - b^r\right)\ov_i^\ell
                    + \sum_{\ell \in Q^r}
                        \left(b^\ell - b^{r + 1}\right) \left(\ox_i^k + \ov_i^\ell - 1\right)
                \right)\\
        &    &=  \left(b^{r+1}-b^r\right)\ox_i^k
                + \sum_{\ell = r + 2}^{\sigma(k)}
                    \left(b^r - b^{r + 1}\right)\ov_i^\ell
                - \left(b^{r + 1} - b^r\right)\ov_i^{r + 1}\\
        &    &\quad +\sum_{\ell \in Q^{r + 1}}
                    \left(b^r - b^{r + 1}\right)\left(\ox_i^k + \ov_i^\ell - 1\right)\label{eq:s-diff-aux}\\
        &    &=  \left(b^{r + 1} - b^r\right)
                \left(
                    \ox_i^k
                    - \sum_{\ell = r + 1}^{\sigma(k)} \ov_i^\ell
                    + \sum_{\ell \in Q^{r + 1}} \left(1 - \ox_i^k - \ov_i^\ell\right)
                \right).
    \end{eqnarray*}

    In \eqref{eq:s-diff-aux}, it has been used that $Q^{r + 1} = Q^r \cup \{ r
    \}$ if $\ox_i^k + \ov_i^r > 1$ and $Q^{r + 1} = Q^r$ otherwise.
\end{proof}

This expression may now be used to prove that $S$ is of the form described at
the start of the section (decreasing, then increasing). This is done by proving
the following result:

\begin{proposition}
    If $0 < r < \sigma(k)$, then $S(r) - S(r - 1) \geq 0$ implies that $S(r + 1)
    - S(r) \geq 0$.
\end{proposition}

\begin{proof}
    Since $b^{r + 1} - b^r > 0$, from
    \[
        S(r + 1) - S(r) =
            \left(b^{r + 1} - b^r\right)
            \left(
                \ox_i^k
                + \sum_{\ell = r + 1}^{\sigma(k)} \ov_i^\ell
                + \sum_{\ell \in Q^{r + 1}} \left(1 - \ox_i^k - \ov_i^\ell\right)
            \right)
    \]
    we infer that $S(r) - S(r - 1) \geq 0$ iff
    \[
        \ox_i^k
        + \sum_{\ell = r}^{\sigma(k)} \ov_i^\ell
        + \sum_{\ell \in Q^r} \left(1 - \ox_i^k - \ov_i^\ell\right)
        \geq 0.
    \]
    Therefore, in order to prove $S(r + 1) - S(r) \geq 0$, it suffices to show
    that
    \[
            \ox_i^k
            + \sum_{\ell = r + 1}^{\sigma(k)} \ov_i^\ell
            + \sum_{\ell \in Q^{r + 1}} \left(1 - \ox_i^k - \ov_i^\ell\right)
        \geq
            \ox_i^k
            + \sum_{\ell = r}^{\sigma(k)} \ov_i^\ell
            + \sum_{\ell \in Q^r} \left(1 - \ox_i^k - \ov_i^\ell\right).
    \]
    However,
    \begin{align*}
        \ox_i^k
        + \sum_{\ell = r + 1}^{\sigma(k)} \ov_i^\ell
            &+ \sum_{\ell \in Q^{r + 1}} \left(1 - \ox_i^k - \ov_i^\ell\right)
             - \left(
                   \ox_i^k
                   + \sum_{\ell = r}^{\sigma(k)} \ov_i^\ell
                   + \sum_{\ell \in Q^r} \left(1 - \ox_i^k - \ov_i^\ell\right)
               \right)\\
            &= \ov_i^r +
                \begin{cases}
                    0,                     &\text{ if } Q^{r + 1} = Q^r\\
                    1 - \ox_i^k - \ov_i^r, &\text{ otherwise}
                \end{cases}\\
            &=  \begin{cases}
                    \ov_i^r,     &\text{ if } Q^{r + 1} = Q^r\\
                    1 - \ov_i^k, &\text{otherwise}
                \end{cases}\\
            &\geq 0.
    \end{align*}
    Which finishes the proof.
\end{proof}

Therefore, $S(r)$ reaches its minimum value for the least $0 < r < \sigma(k)$
such that $S(r) - S(r - 1) \leq 0$ and $S(r + 1) - S(r) > 0$, if it exists; $r =
0$ if $S$ is strictly increasing, or $r = \sigma(k)$ if it is strictly
decreasing or constant. This characterisation lends itself to a simple algorithm
for finding this $r$: simply iterate from $r = 1$ to $\sigma(k) - 1$ until the
minimality condition is satisfied.

However, continuing this study of the difference $S(r + 1) - S(r)$, a simple
rule for easily concluding $S(r) - S(r - 1) \leq 0$ may be obtained without
having to compute
$
    \sum_{\ell \in Q^{r + 1}} \left(1 - \ox_i^k - \ov_i^\ell\right)
$:

\begin{lemma}
    If
    $
        \ox_i^k \leq \sum_{\ell = 1}^{\sigma(k)} \ov_i^\ell
    $,
    then $S(r) - S(r - 1) \leq 0$.
\end{lemma}

\begin{proof}
    It is enough to note that $b^{r + 1} - b^r$ is always positive and
    $
        \sum_{\ell \in Q^{r + 1}}\left(1 - \ox_i^k - \ov_i^\ell\right)
    $
    always less than or equal to zero.
\end{proof}

\subsection{Separation algorithm} % ............................................
\label{ssc:las:soi:separation-algorithm}

Last section presented a method for solving minimisation problem
\eqref{eq:minimisation-problem}. As indicated in Subsection
\ref{ssc:las:soi:introduction}, it may now be checked whether the restriction
resulting from this minimisation procedure is violated by \fracsol. If that were
the case, it would be added to \slla. This procedure is resumed in the following
algorithm:

\begin{algorithm}
    \DontPrintSemicolon
    \KwData{
	\fracsol~an optimal fractional solution of the linear relaxation of
	\slla.
    }
    \KwResult{
	An inequation from family \ref{eq:sll1-inequalities} separating
	\fracsol~ if it can be found, nothing otherwise.
    }
    
    \Begin{
        \For{$k \in K$} {
            \For{$\is$} {
                $r_i^k = 0$\;
                \While{
                    $r_i^k < \sigma(k)$
                    \rm{ and }
                    $\ox_i^k \leq \sum_{\ell=r_i^k}^k \ov_i^\ell$
                } {
                    $r_i^k = r_i^k + 1$\;
                }
                \While{
                    $r_i^k < \sigma(k)$
                    \rm{ and }
                    $S(r_i^k+1) - S(r_i^k) \leq 0$
                } {
                    $r_i^k = r_i^k + 1$\;
                }

                $
                    Q_i^k = \{
                        \ell \in \{1, \ldots, r_i^k - 1\} \mid \ox_i^k + \ov_i^\ell > 1
                    \}
                $
            }
            Return constraint
            \[
                z^k \leq \sum_\is \left(
                    b^{r_i^k}x_i^k
                    + \sum_{\ell=r_i^k+1}^{\sigma(k)}
                        \left(b^\ell - b^{r_i^k}\right) v_i^\ell
                    + \sum_{\ell\in Q_i^k}
                        \left(b^\ell - b^{r_i^k}\right) \left(x_i^k+v_i^\ell-1\right)
                \right)
           \]
           \noindent
           iff it is violated by \fracsol.
        }
    }
    \caption{Separation of inequalities for \slla.\label{alg:separation-1}}
\end{algorithm}

\begin{algorithm}
    \DontPrintSemicolon
    \KwData{
	$(\ox_i^k, \ov_i^\ell, \oz_i^k)$~an optimal fractional solution of the
	linear relaxation of \sllb.
    }
    \KwResult{
	An inequation from family \ref{eq:sll2-inequalities} separating
	\fracsol~ if it can be found, nothing otherwise.
    }
    
    \Begin{
        \For{$k \in K$} {
            \For{$\is$} {
                $r_i^k = 0$\;
                \While{
                    $r_i^k < \sigma(k)$
                    \rm{ and }
                    $\ox_i^k \leq \sum_{\ell=r_i^k}^k \ov_i^\ell$
                } {
                    $r_i^k = r_i^k + 1$\;
                }
                \While{
                    $r_i^k < \sigma(k)$
                    \rm{ and }
                    $S(r_i^k+1) - S(r_i^k) \leq 0$
                } {
                    $r_i^k = r_i^k + 1$\;
                }

                $
                    Q_i^k = \{
                        \ell \in \{1, \ldots, r_i^k - 1\} \mid \ox_i^k + \ov_i^\ell > 1
                    \}
                $
            }
            Return constraint
            \[
                z_i^k \leq 
                    b^{r_i^k}x_i^k
                    + \sum_{\ell=r_i^k+1}^{\sigma(k)}
                        \left(b^\ell - b^{r_i^k}\right) v_i^\ell
                    + \sum_{\ell\in Q_i^k}
                        \left(b^\ell - b^{r_i^k}\right)\left(x_i^k+v_i^\ell-1\right)
           \]
           \noindent
           iff it is violated by $(\ox_i^k, \ov_i^\ell, \oz_i^k)$.
        }
    }
    \caption{Separation of inequalities for \sllb.\label{alg:separation-2}}
\end{algorithm}


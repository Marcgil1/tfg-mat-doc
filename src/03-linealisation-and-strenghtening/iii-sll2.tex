\section{Second single level linear formulation} % ----------------------------
\label{sec:las:sll2}

\subsection{Formulation} % .....................................................
\label{ssc:las:sll2:formulation}

The set of variables presented in the last section can be modified in order to
reach an alternative formulation. The idea of dividing the total benefit as the
sum of partial benefits remains intact. However, the degree of granularity in
the election of variables may be increased. Concretely, for each client $\kk$
and product $\is$ variable $z_i^k$ will be defined as the benefits obtained from
the acquisition of $i$ by $k$ if this acquisition takes place, and as zero if
$k$ does not buy $i$. The objective function will thus be
\[
    \max_{v,x,z} \sum_\kk \sum_\is z_i^k.
\]

\noindent
Note that
$
    z^k = \sum_\is z_i^k, \forall\kk
$.
The restrictions to add will be
{
    \newcommand{\suml}    {\sum_{\ell=1}^{\sigma(k)}}

    \begin{eqnarray}
        z_i^k\leq \suml b^\ell v_i^\ell,&\forall\kk,\is \label{eq:sll2-prev:a}\\
        z_i^k\leq b^{\sigma(k)} x_i^k,  &\forall\kk,\is \label{eq:sll2-prev:b}\\
        z_i^k\geq 0,                    &\forall\kk,\is.\label{eq:sll2-prev:c}
    \end{eqnarray}
}

That for any $\kk$ and $\is$ these restrictions conform to the definition of
$z_i^k$ given above is justified by the following case analysis:
\begin{itemize}
    
    \item % k sí compra i
        If $k$ buys $i$, then \eqref{eq:sll2-prev:b} is redundant:
        $
            z_i^k \leq b^{\sigma(k)}x_i^k = b^{\sigma(k)}
        $
	and \eqref{eq:sll2-prev:a} gives the price of the $i$'th product:
        {
            \newcommand{\suml}    {\sum_{\ell=1}^{\sigma(k)}}
            $
                z_i^k \leq \suml b^\ell v_i^\ell.
            $
        }
    \item % k no compra i
	If $k$ does not buy $i$, then \eqref{eq:sll2-prev:b} takes the form
        $
            z_i^k \leq b^{\sigma(k)}x_i^k = 0
        $
        and \eqref{eq:sll2-prev:a} is redundant:
        {
            \newcommand{\suml}    {\sum_{\ell=1}^{\sigma(k)}}
            $
                z_i^k \leq \suml b^\ell v_i^\ell.
            $
        }
    
\end{itemize}

The addition of this set of variables to \slnl gives then rise to the following
formulation, which we will refer to as \sllb,

{
    \newcommand{\bover}   {\overline{B(k, i)}}
    \newcommand{\bne}     {\bover\neq\varnothing}
    \newcommand{\sumk}    {\sum_{\kk}}
    \newcommand{\sumi}    {\sum_{\is}}
    \newcommand{\sumlm}   {\sum_{\ell=1}^{M}}
    \newcommand{\sumj}    {\sum_{j\in\bover}}
    \newcommand{\sumls}   {\sum_{\ell=1}^{\sigma(k)}}
    \newcommand{\sumlsm}  {\sum_{\ell=\sigma(k)+1}^{M}}
    
    \begin{eqnarray}
        &\max_{v, x, z}
            & \sumk\sumi z^k                                           \label{eq:sll2:a}\\
        &\text{s.t.}
            & \sumi  x_i^k                  \leq 1, \quad\forall\kk    \label{eq:sll2:b}\\
        &   & \sumlm v_i^\ell               \leq 1, \quad\forall\ii    \label{eq:sll2:c}\\
        &   & \sumj x_j^k + \sumls v_i^\ell \leq 1, \quad\forall\kk,\is:\bne
                                                                       \label{eq:sll2:d}\\
        &   & x_i^k + \sumlsm v_i^\ell      \leq 1, \quad\forall\kk,\is\label{eq:sll2:e}\\
        &   & z^k \leq \sumls b^\ell v_i^\ell,      \quad\forall\kk,\is\label{eq:sll2:f}\\
        &   & z^k \leq b^{\sigma(k)} x_i^k,         \quad\forall\kk,\is\label{eq:sll2:g}\\
        &   & v_i^\ell, x_i^k \in \binset, z^k\geq 0\quad\forall\kk,\is,\lm
                                                                       \label{eq:sll2:h}
    \end{eqnarray}
}

\subsection{Inequalities} % ....................................................
\label{ssc:las:sll2:inequalities}

In a similar manner to formulation \slla, the families \eqref{eq:sll2:f} and
\eqref{eq:sll2:g} from \sllb lead to poor linear relaxations. In order to
alleviate this issue, the authors of \cite{ca:rpp} defined the following family
of valid inequalities:

\begin{proposition}
    For any customer $\kk$, product $\is$, integer
    $r_i^k\in\{0,\ldots,\sigma(k)\}$ and subset $Q_i^k\subseteq\{1, \ldots,
    r_i^k-1\}$, the following is a valid inequality for formulation \sllb:

    \begin{equation}
        z^k \leq
            b^{r_i^k}x_i^k
            + \sum_{\ell=r_i^k+1}^{\sigma(k)}
                \left(b^\ell - b^{r_i^k}\right) v_i^\ell
            + \sum_{\ell\in Q_i^k}
                \left(b^\ell - b^{r_i^k}\right)\left(x_i^k+v_i^\ell-1\right).
        \label{eq:sll2-inequalities}
    \end{equation}
\end{proposition}

\begin{proof}
    As in the case of the family of inequations \eqref{eq:sll1-inequalities},
    the validity of \eqref{eq:sll2-inequalities} can be proved by a case
    analysis on $x_i^k$. If $x_i^k = 1$, then $v_i^{\ell_0} = 1$ for some
    $\ell_0 \leq \sigma(k)$, $i$ must have had an assigned price within $k$'s
    budget. We will consider two cases for $\ell_0$:
    \begin{enumerate}
        
        \item % \ell_0 \leq r_i^k
	    If $\ell_0 \leq r_i^k$, then $v_i^\ell = 0$ for all $\ell > r_i^k$,
	    so
	    $
	        \sum_{\ell = r_i^k + 1}^{\sigma(k)}
	            \left(b^\ell - b^{r_i^k}\right)v_i^\ell
	        = 0
	    $,
	    and the right-hand side results
	    \begin{equation}
	        b^{r_i^k}
	        + \sum_{\ell \in Q_i^k}
	            \left(b^\ell - b^{r_i^k}\right) v_i^\ell.
	        \label{eq:sll2-inequalities:aux}
	    \end{equation}
	    If $\ell_0 \in Q_i^k$, then \eqref{eq:sll2-inequalities:aux} is
	    equal to
	    $
	        b^{r_i^k} + b^{\ell_0} - r^{r_i^k} = b^{\ell_0}
	    $,
	    which leads to the trivially true condition $z^k \leq b^{\ell_0}$.
	    If $\ell_0 \not\in Q_i^k$, then \eqref{eq:sll2-inequalities:aux}
	    equals $b^{r_i^k}$, which is greater than $\ell_0$ due to the
	    relation $\ell_0 \leq r_i^k$, and the inequality is again true.
        \item % \ell_0 >    r_i^k
            If $\ell_0 > r_i^k$, then $v_i^\ell = 0$ for all
            $
                \ell \in Q_i^k \subseteq \{ 1, \ldots, r_i^k - 1 \}
            $
            and the right-hand side results
            \[
                b^{r_i^k} + b^{\ell_0} - b^{r_i^k} = b^{\ell_0}.
            \]

    \end{enumerate}
    In any case, if $x_i=0$, the right-hand side of \eqref{eq:sll2-inequalities}
    is greater than or equal to the upper bound of $z_i^k$, $b^{\ell_0}$. If, on
    the other side, $x_i^k = 0$, the last argument used in the proof of the
    validity of \ref{eq:sll1-inequalities:aux} is usable, and implies that the
    right-hand side of \eqref{eq:sll2-inequalities} is non-negative. Since in
    this case $z_i^k \leq 0$ is a valid upper bound, the non-negativity of
    \eqref{eq:sll2-inequalities}'s right-hand side implies again that
    \eqref{eq:sll2-inequalities} is a valid inequality.
\end{proof}

As in Section \ref{ssc:las:sll1:inequalities}, the inequation family defined in
the last result is expressive enough as to contain both restriction sets for
$z_i^k$ on \sllb. Indeed, if $\kk$ and $\is$, then taking $r_i^k = 0$ and $Q_i^k
= \varnothing$, \eqref{eq:sll2-inequalities} takes the form of
\eqref{eq:sll2:f}:
\begin{eqnarray*}
    z^k&\leq&
        b^{r_i^k}x_i^k
        + \sum_{\ell=r_i^k+}^{\sigma(k)}
            \left(b^\ell - b^{r_i^k}\right) v_i^\ell
        + \sum_{\ell\in Q_i^k}
            \left(b^\ell - b^{r_i^k}\right) \left(x_i^k+v_i^\ell-1\right)\\
       &\stackrel{(Q_i^k = \varnothing)}{\iff}& z_i^k \leq
           b^{r_i^k} x_i^k
           + \sum_{\ell = r_i^k + 1}^{\sigma(k)}
               \left(b^\ell - b^{r_i^k}\right)v_i^\ell\\
       &\stackrel{(r_i^k = 0)}{\iff}& z_i^k \leq
           \sum_{\ell = 1}^{\sigma(k)}
               b^\ell v_i^\ell
\end{eqnarray*}

In a similar manner, if $\kk$ and $\is$, we may take $r_i^k = \sigma(k)$ and
$Q_i^k = \varnothing$ in order to transform \eqref{eq:sll2-inequalities} into
\eqref{eq:sll2:g}:
\begin{eqnarray*}
    z^k&\leq&
        b^{r_i^k}x_i^k
        + \sum_{\ell=r_i^k+}^{\sigma(k)}
            \left(b^\ell - b^{r_i^k}\right) v_i^\ell
        + \sum_{\ell\in Q_i^k}
            \left(b^\ell - b^{r_i^k}\right) \left(x_i^k+v_i^\ell-1\right)\\
       &\stackrel{(Q_i^k = \varnothing)}{\iff}& z_i^k \leq
           b^{r_i^k} x_i^k
           + \sum_{\ell = r_i^k + 1}^{\sigma(k)}
               \left(b^\ell - b^{r_i^k}\right) v_i^\ell\\
       &\stackrel{(r_i^k = \sigma(k))}{\iff}& z_i^k \leq
           b^{\sigma(k)} x_i^k
\end{eqnarray*}

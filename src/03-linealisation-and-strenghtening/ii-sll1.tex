\section{First single level linear formulation} % ------------------------------
\label{sec:las:sll1}

\subsection{Formulation} % .....................................................
\label{ssc:las:sll1:formulation}

The function to linearise is the sum of the prices of all sold products.  One
way of proceeding is defining a new set of variables over which
\eqref{eq:slnl:a} is a linear function. In this case, we will use, for each
client $k$, $z^k$, the benefit obtained from client $k$. Thus, the objective may
be expressed as the sum of the benefits obtained from each client. i.e.,

\[
    \max_{v,x,z} \sum_\kk z^k
\]

Several restrictions will be added to the formulation that will relate $z^k$
with the old variables in order to codify $z^k$'s meaning. These restrictions
will be the following:

{
    \newcommand{\suml} {\sum_{\ell=1}^{\sigma(k)}}
    \newcommand{\sumi} {\sum_\is}
    
    \begin{eqnarray}
        z^k &\leq& \suml b^\ell v_i^\ell + b^{\sigma(k)}\left(1-x_i^k\right),
            \quad \forall\kk, \is \label{eq:sll1-prev:a}\\
        z^k &\leq& b^{\sigma(k)} \sumi x_i^k
            \quad \forall\kk      \label{eq:sll1-prev:b}\\
        z^k &\geq& 0
            \quad \forall\kk.     \label{eq:sll1-prev:c}
    \end{eqnarray}
}

These inequalities codify $z^k$'s meaning in the following manner: If $k$ cannot
acquire any product in its preference list, then the benefit from him obtained
will be zero. If $k$ acquires the (unique) product $i$, the benefit from him
obtained will be $i$'s price,
$
    \sum_{\ell=1}^{\sigma(k)} b^\ell v_i^\ell
$.
More concretely,
\begin{enumerate}
    
    \item % k no compra
        If $k$ acquires no product, then
        $
            \sum_\is x_i^k = 0
        $,
	and \eqref{eq:sll1-prev:b} implies then $z^k \leq 0$, the desired value of
	$z^k$ in this case. Furthermore, for all $\is$,
	\[
	    z^k \leq 0 \leq \sum_{\ell=1}^{\sigma(k)} b^\ell v_i^\ell + b^{\sigma(k)}
	\]
	and thus restrictions \eqref{eq:sll1-prev:a} do not change $z^k$'s
	meaning.
    \item % k sí compra
	If $k$ acquires product $i_0 \in S^k$, then
	$b^{\sigma(k)}\left(1-x_{i_0}^k\right)=0$, and \eqref{eq:sll1-prev:a} leads to
	\[
	    z^k \leq \sum_{\ell=1}^{\sigma(k)} b^\ell v_{i_0}^\ell.
	\]
	Thus, $z^k$ is less than or equal to the price assigned to $i_0$. For
	any other $\is, i \neq i_0$, the restriction \eqref{eq:sll1-prev:a} is
	redundant
	$
	    z^k \leq \sum_{\ell=1}^{\sigma(k)} b^\ell v_i^\ell + b^{\sigma(k)}
	        =    b^{\sigma(k)}
	$,
	as well as \eqref{eq:sll1-prev:b}:
	$
	    z^k \leq b^{\sigma(k)}\sum_\is x_i^k = b^{\sigma(k)}.
	$
	
\end{enumerate}

These constraints would suffice for defining the new formulation. However, a
further improvement must be made on them by noting that \eqref{eq:sll1-prev:a}
can be reinforced as follows:

\begin{proposition}
    For any $\kk$, the following is a reinforcement of \eqref{eq:sll1-prev:b}:
    {
        \newcommand{\suml}    {\sum_{\ell=1}^{\sigma(k)}}
        \newcommand{\sumi}    {\sum_{\is}}
        \newcommand{\sumjneq} {\sum_{j \in S^k : j \neq i}}

        \begin{equation}
            z^k \leq \suml b^\ell v_i^\ell + b^{\sigma(k)} \sumjneq x_j^k,
                \quad \forall \kk, \is. \label{eq:sll1-prev:d}
        \end{equation}
    }
\end{proposition}

\begin{proof}
    The proof is done by a case analysis:
    \begin{itemize}
            
        \item % k no compra ningún i
            If $k$ buys no $i' \in S^k$, then
            {
                \newcommand{\sumjneq} {\sum_{j \in S^k : j \neq i}}
                $\sumjneq x_j^k = 0$
            }
            and the restriction results
            $
                z^k \leq \sum_{\ell=1}^{\sigma(k)} b^\ell v_i^\ell
            $,
	    thus being redundant, for in this case, \eqref{eq:sll1-prev:b} takes
	    the form $z^k \leq 0$.
        \item % k compra i_0 \neq i
            If $k$ buys some $i_0 \neq i$, then
            {
                \newcommand{\sumjneq} {\sum_{j \in S^k : j \neq i}}
                $\sumjneq x_j^k = 1$
            }
            and the restriction results
            $
                z^k \leq
                    \sum_{\ell=1}^{\sigma(k)} b^\ell v_i^\ell
                    + b^{\sigma(k)}
            $,
            as the corresponding one in \eqref{eq:sll1-prev:a}.
        \item % k compra precisamente i
            If $k$ buys precisely $i$, the restriction results
            $
                z^k \leq \sum_{\ell=1}^{\sigma(k)} b^\ell v_i^\ell
            $,
            as the corresponding one in \eqref{eq:sll1-prev:a}.
    \end{itemize}
\end{proof}

Finally, the new objective function, variables, and constraints can be added to
\slnl in order to obtain the following formulation, named \slla:
{
    \newcommand{\bover}   {\overline{B(k, i)}}
    \newcommand{\bne}     {\bover\neq\varnothing}
    \newcommand{\sumk}    {\sum_{\kk}}
    \newcommand{\sumi}    {\sum_{\is}}
    \newcommand{\sumlm}   {\sum_{\ell=1}^{M}}
    \newcommand{\sumj}    {\sum_{j\in\bover}}
    \newcommand{\sumls}   {\sum_{\ell=1}^{\sigma(k)}}
    \newcommand{\sumlsm}  {\sum_{\ell=\sigma(k)+1}^{M}}
    \newcommand{\sumjneq} {\sum_{j \in S^k : j \neq i}}
    
    \begin{eqnarray}
        &\max_{v, x, z}
             & \sumk z^k                                                \label{eq:sll1:a}\\
        &\text{s.t.}
             & \sumi  x_i^k                  \leq 1, \quad\forall\kk    \label{eq:sll1:b}\\
        &    & \sumlm v_i^\ell               \leq 1, \quad\forall\ii    \label{eq:sll1:c}\\
        &    & \sumj x_j^k + \sumls v_i^\ell \leq 1, \quad\forall\kk,\is:\bne
                                                                        \label{eq:sll1:d}\\
        &    & x_i^k + \sumlsm v_i^\ell      \leq 1, \quad\forall\kk,\is\label{eq:sll1:e}\\
        &    & z^k \leq \sumls b^\ell v_i^\ell + b^{\sigma(k)}\sumjneq x_j^k,
                                                     \quad\forall\kk,\is\label{eq:sll1:f}\\
        &    & z^k \leq b^{\sigma(k)}\sumi x_i^k,    \quad\forall\kk    \label{eq:sll1:g}\\
        &    & v_i^\ell, x_i^k \in \binset, z^k\geq 0\quad\forall\kk,\is,\lm
                                                                       \label{eq:sll1:h}
    \end{eqnarray}
}

\subsection{Inequalities} % ....................................................
\label{ssc:las:sll1:inequalities}

As already noted, \slla is a valid formulation for the Rank Pricing Problem.
However, the newly introduced families of inequations, \eqref{eq:sll1:f} and
\eqref{eq:sll1:g}, are weak restrictions on $z^k$ that may give poor linear
relaxations during the execution of the branch-and-bound procedure. To see this,
note that both right hand sides of \eqref{eq:sll1:f} and \eqref{eq:sll1:g}
contain a sum of $x_i^k$ variables multiplied by the constant quantity
$b^{\sigma(k)}$, which may be quite big. Therefore, if for some instance
$z^{k_0} = 0$ for some client $k_0$, then the $x_i^{k_0}$ variables may take any
value whatsoever.

In order to alleviate this issue, the authors of \cite{ca:rpp} defined the
following set of valid inequalities for formulation \slla, which impose further
bounds on the variables:

\begin{proposition}
    For any customer $\kk$, integers $r_i^k\in\{0,\ldots,\sigma(k)\}~\forall\is$
    and subsets $Q_i^k\subseteq\{1,\ldots,r_i^k-1\}~\forall\is$, the following
    is a valid inequality for problem \slla:

    \begin{equation}
        z^k \leq \sum_\is \left(
            b^{r_i^k}x_i^k
            + \sum_{\ell=r_i^k+1}^{\sigma\left(k\right)}
                \left(b^\ell - b^{r_i^k}\right) v_i^\ell
            + \sum_{\ell\in Q_i^k}
                \left(b^\ell - b^{r_i^k}\right) \left(x_i^k+v_i^\ell-1\right)
        \right).
        \label{eq:sll1-inequalities}
    \end{equation}
\end{proposition}

\begin{proof}
    The proof proceeds by analysing two cases, proving for each of them that the
    right-hand side of \eqref{eq:sll1-inequalities} takes a value greater or
    equal to that of a valid restriction of $z^k$ (\eqref{eq:sll1:f} or
    \eqref{eq:sll1:g}).

    Let us suppose that $x_{i_0}^k=1$ for some $i_0 \in S^k$. That is, that $k$
    buys product $i$, and focus on the summand of \eqref{eq:sll1-inequalities}
    corresponding to $i_0$:

    \begin{equation}
        b^{r_{i_0}^k}x_{i_0}^k
            + \sum_{\ell = r_{i_0}^k + 1}
                \left(
                    b^\ell - b^{r_{i_0}^k}
                \right) v_{i_0}^\ell
            + \sum_{\ell \in Q_{i_0}^k  }
                \left(
                    b^\ell - b^{r_{i_0}^k}
                \right)
                \left(
                    x_{i_0}^k + v_{i_0}^\ell - 1
                \right).
        \label{eq:sll1-inequalities:aux}
    \end{equation}

    Since $i_0$ must be within $k$'s budget, there exists some $\ell_0 \leq
    \sigma(k)$ such that $v_{i_0}^{\ell_0} = 1$. It will be proved by a further
    case analysis on $\ell_0$ that \eqref{eq:sll1-inequalities:aux} is greater
    than or equal to the price assigned to $i_0$, $b^{\ell_0}$.
    \begin{enumerate}
        
        \item % \ell_0 >  r_{i_0}^k
            If $\ell_0 > r_{i_0}^k$, then $v_{i_0}^\ell = 0$ for every
            $
                \ell \in Q_{i_0}^k \subseteq \{1, \ldots, r_{i_0}^k - 1\}
            $,
            so
            $
                \sum_{\ell \in Q_{i_0}^k}
                    \left(b^\ell - b^{r_{i_0}^k}\right)\left(x_{i_0}^k + v_{i_0}^\ell - 1\right) = 0
            $
            and, since a product has no more than one price,
            $
                \sum_{\ell = r_i^k + 1}^{\sigma\left(k\right)}
                    \left(b^\ell - b^{r_{i_0}^k}\right)v_{i_0}^\ell
                = b^{\ell_0} - b^{r_{i_0}^k}.
            $
            Finally, \eqref{eq:sll1-inequalities:aux} reduces thus to
            $
                b^{r_{i_0}^k} + b^{\ell_0} - b^{r_{i_0}^k} = b^{\ell_0}.
            $
            
        \item % \ell_0 <= r_{i_0}^k
	    If, on the contrary, $\ell_0 \leq r_{i_0}^k$, then $v_{i_0}^\ell =
	    0$ for any $\ell > r_{i_0}^k$, and the first sum is now the one
	    which vanishes,
            $
                \sum_{\ell = r_{i_0}^k + 1}^{\sigma(k)}
                    \left(b^\ell - b^{r_{i_0}^k}\right)v_{i_0}^\ell
                = 0
            $,
            so that \eqref{eq:sll1-inequalities:aux} gets reduced to
            \[
                b^{r_{i_0}^k}
                    + \sum_{\ell \in Q_{i_0}^k}
                        (b^\ell - b^{r_{i_0}^k}) v_{i_0}^k.
            \]
            If $\ell_0 \in Q_{i_0}^k$, this quantity is
            $
                b^{r_{i_0}^k} + b^\ell - b^{r_{i_0}^k} = b^\ell
            $.
	    If $\ell \not\in Q_{i_0}^k$, then \eqref{eq:sll1-inequalities:aux}
	    equals $b^{r_{i_0}^k}$, which is larger than $b^{\ell_0}$, for we
	    are supposing that $\ell_0 \leq r_{i_0}^k$.
        
    \end{enumerate}

    In any case, if $k$ buys some $i_0$, the corresponding summand of $i_0$ in
    the right-hand side of \eqref{eq:sll1-inequalities}, and therefore all of the
    right-hand side, is greater than or equal to the maximum value of $z^k$
    allowed by restriction \eqref{eq:sll1:f}. Thus, \eqref{eq:sll1-inequalities} is
    a valid inequality in this case.

    Let us now suppose that $x_i^k = 0$ for all $\is$; that is, $k$ buys no
    product. Consider the summand of \eqref{eq:sll1-inequalities} associated to an
    arbitrary product $\is$:
    \[
        \sum_{\ell = r_i^k + 1} \left(b^\ell - b^{r_i^k}\right)v_i^\ell
            + \sum_{\ell \in Q_i^k} \left(b^\ell - b^{r_i^k}\right)\left(v_i^\ell - 1\right).
    \]

    Note that the term $b^{r_i^k}x_i^k$ has been removed, since $x_i^k = 0$. As
    in the first summand $\ell \geq r_i^k + 1$, $b^\ell - b^{r_i^k} > 0$, so
    $
        \sum_{\ell = r_i^k + 1}^{\sigma\left(k\right)}
            \left(b^\ell - b^{r_i^k}\right)v_i^\ell \geq 0
    $.
    As in the second summand $\ell < r_i^k$, $b^\ell - b^{r_i^k} < 0$, and
    $v_i^\ell - 1 \leq 0$, so
    $
        \sum_{\ell \in Q_i^k} \left(b^\ell - b^{r_i^k}\right)\left(v_i^\ell - 1\right) \geq 0
    $.
    Therefore, both summands of \eqref{eq:sll1-inequalities} are, in this case,
    non-negative, so \eqref{eq:sll1-inequalities} is a non-negative upper bound
    on $z^k$, and thus a valid inequality for $z^k$ also in the case $x_i^k = 0,
    \forall\is$.
\end{proof}

It is a remarkable fact that the given family of inequalities is expressive
enough to contain both sets of constraints over $z^k$ present in \slla. Indeed,
constraints \eqref{eq:sll1:f}
\[
    z^k \leq
        \sum_{\ell = 1}^{\sigma(k)}  v^\ell v_i^\ell
        + \sum_{j \in S^k: j \neq i} x_j^k
    , \quad \forall \kk, \is
\]
can be obtained from \eqref{eq:sll1-inequalities} by setting $r_i^k = 0$,
$
    r_j^k = \sigma(k), \forall j \in S^k-\{i\}
$
and
$
    Q_j^k = \varnothing, \forall j \in S^k
$,
as the following derivation implies:

\begin{eqnarray*}
    z^k&\leq& \sum_{j \in S^k} \left(
            b^{r_j^k}x_j^k
            + \sum_{\ell = r_j^k + 1}^{\sigma\left(k\right)} \left(b^\ell - b^{r_j^k}\right) v_j^\ell
            + \sum_{\ell \in Q_j^k} \left(b^\ell - b^{r_j^k}\right) \left(x_j^k + v_j^\ell - 1\right)
        \right)\\
       &\iff& z^k \leq
            b^{r_i^k}x_i^k
            + \sum_{\ell = r_i^k + 1}^{\sigma\left(k\right)} \left(b^\ell - b^{r_i^k}\right) v_i^\ell
            + \sum_{\ell \in Q_i^k} \left(b^\ell - b^{r_i^k}\right) \left(x_i^k + v_i^\ell-1\right)\\
       &    & \quad\quad\quad\quad + \sum_{j \in S^k: j \neq i} \left(
                b^{r_j^k}x_j^k
                + \sum_{\ell = r_j^k + 1}^{\sigma\left(k\right)} \left(b^\ell-b^{r_j^k}\right)v_j^\ell
                + \sum_{\ell \in Q_j^k} \left(b^\ell - b^{r_j^k}\right) \left(x_i^k +v_i^\ell-1\right)
            \right)\\
       &\stackrel{\left(r_i^k = 0\right)}{\iff}& z^k \leq
            \sum_{\ell = r_i^k + 1}^{\sigma\left(k\right)} b^\ell v_i^\ell
            + \sum_{\ell \in Q_i^k} b^\ell \left(x_i^k + v_i^\ell - 1\right)\\
       && \quad\quad     
            + \sum_{j \in S^k: j \neq i} \left(
                \sum_{\ell = r_j^k + 1}^{\sigma\left(k\right)} \left(b^\ell - b^{r_i^k}\right)v_j^\ell
                + \sum_{\ell \in Q_j^k} \left(b^\ell - b^{r_i^k}\right)\left(x_i^k + v_i^\ell-1\right)
            \right)\\
       &\stackrel{\left(r_j^k = \sigma\left(k\right)\right)}{\iff}& z^k \leq
            \sum_{\ell = r_i^k + 1}^{\sigma\left(k\right)} b^\ell v_i^\ell
            + \sum_{\ell \in Q_i^k} b^\ell \left(x_i^k + v_i^\ell - 1\right)
            + \sum_{j \in S^k: j \neq i} \left(
                b^\sigma\left(k\right) x_j^k
                + \sum_{\ell \in Q_j^k} \left(b^\ell-b^{\sigma\left(k\right)}\right) \left(x_i^k + v_i^\ell - 1\right)
            \right)\\
       &\stackrel{\left(Q_j^k = \varnothing\right)}{\iff}& z^k \leq
           \sum_{\ell = 1}^{\sigma\left(k\right)} b^\ell v_i^\ell
           + b^{\sigma\left(k\right)} \sum_{j \in S^k: j \neq i} x_j^k.
\end{eqnarray*}

Similarly, inequalities \eqref{eq:sll1:g} can be derived from
\eqref{eq:sll1-inequalities} by setting $r_i^k = \sigma(k)$ and $Q_i^k =
\varnothing, \forall\is$, as evidenced by the following derivation:

\begin{eqnarray*}
    z^k&\leq \sum_{j \in S^k} \left(
            b^{r_j^k}x_j^k
            + \sum_{\ell = r_j^k + 1}^{\sigma\left(k\right)} \left(b^\ell - b^{r_j^k}\right) v_j^\ell
            + \sum_{\ell \in Q_j^k} \left(b^\ell - b^{r_j^k}\right) \left(x_j^k + v_j^\ell - 1\right)
        \right)\\
       &\stackrel{\left(r_i^k = \sigma\left(k\right)\right)}{\iff} z^k \leq
           \sum_\is \left(
               b^{\sigma\left(k\right)} x_i^k
               + \sum_{\ell \in Q_i^k} \left(b^\ell - b^{r_i^k}\right)\left(x_i^k + v_i^\ell-1\right)
           \right)\\
       &\stackrel{\left(Q_i^k = \varnothing\right)}{\iff} z^k \leq
           b^{\sigma\left(k\right)} \sum_\is x_i^k
\end{eqnarray*}

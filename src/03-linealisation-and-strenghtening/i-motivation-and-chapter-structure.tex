\section{Motivation and chapter structure} % -----------------------------------
\label{sec:las:motivation}

We have already said in the introduction that, despite being a highly expressive
paradigm that allows solving a wide variety of problems, integer programming is
computationally expensive. Thus, two formulations for the very same problem may
give rise to completely different solving times. This was the reason for
abandoning formulation \bnlv, which had a conceptually simple structure
explicitly reflecting the two agents (seller and client) involved in the \rpp,
in favour of \slnl, which was computationally simpler due to being a
single-level formulation. This chapter's objective is to continue presenting
results in this line, following \cite{ca:rpp}, deriving from \slnl two further
formulations dealing with its main caveat: non-linearity.

Being all restrictions in \slnl linear inequalities, the only element
introducing non-linearity is the objective function \eqref{eq:slnl:a}, in which
$x_i^k$ variables get multiplied with $v_i^\ell$. This is thus the element to
improve; improvement done by defining a new set of variables allowing a linear
expression for the cost function. Two such sets of variables will be presented
in subsections \ref{ssc:las:sll1:formulation} and
\ref{ssc:las:sll2:formulation}, each one leading to a different formulation.

The new linear formulations will still leave room for improvement, their
restrictions being too weak. The next results will thus be two families of valid
restrictions \eqref{ssc:las:sll1:inequalities} and
\eqref{ssc:las:sll2:inequalities}, which will be, however, too large to be
directly added to the formulations. Thus, in Section \ref{sec:las:separation},
an algorithm will be presented for dynamically adding them to the formulation
during the problem's resolution, following the branch-and-cut schema outlined in
the introductory chapter.

\chapter*{Abstract (English)} % ------------------------------------------------

The object of study of the present work is the Rank Pricing Problem (RPP), an
optimisation problem in which some products must me priced so as to maximise
their seller's gains. Now, we will present the RPP, and provide the reader a
summary of the structure of this work.

\paragraph*{}

The Rank Pricing Problem presupposes the existence of a seller and of a set of
clients. The seller offers various goods and wishes to prize them so as to
maximise his benefits. In turn, each client is characterised by a list of
preferences. These lists consist of products from those offered by the seller.
Products in them are totally ordered, so that for each pair of distinct products
$i_1$ and $i_2$ in the preference list of some customer $k$, either $i_1$ is
strictly better than $i_2$ for $k$, or $i_2$ is strictly better than $i_1$ for
$k$.  Furthermore, each client has a budget, which is the maximum amount of
money he can afford to expend. In this setting, each client wishes to purchase
exactly one product of maximum preference for him, and whose prices be less than
or equal to his budget.

The RPP has been treated in depth in the article \cite{ca:rpp}. Most of the
results of chapters 2, 3 and 4 have been extracted from this source. Unless
otherwise stated, all results should be considered to have been extracted from
the article of Calvete et al.

This work is divided in six chapters, whose contents we will now summarise.

\paragraph*{}

The present work requires several definitions and results from three different
areas of optimisation. Namely linear, integer, and bilevel programming. These
are presented in the first chapter.

An understanding of linear programming is a necessary requirement for
understanding the integer programming models which conform the core of this
work. For this reason, it is treated in the first section of the introduction.
After giving a short historical overview of this paradigm, several of its
elementary definitions are presented. Furthermore, several results from duality
theory are shortly presented. These are needed in the second chapter, where they
are used to derive a single-level formulation from the bilevel formulations
initially defined.

Integer programming is introduced in the second section of the introduction.
This paradigm is used to define most formulations in this work. Thus, several of
its elementary definitions are presented, as well as comments on its
computational complexity and on two of its resolution methods: branch and bound,
and branch and cut. Furthermore, we introduce several results on totally
unimodular matrices, which allow solving some integer problems as linear ones.
These results are used in the second chapter.

Finally, bilevel programming is presented in the third section of the
introduction. This branch of optimisation studies the definition of problems
with a bilevel structure. On the first level, one wishes to solve a linear or
integer programming problem. Furthermore, amongst the restrictions of this first
level problem stands the requirement that some variables be solution to a second
problem. Bilevel programming permits giving some interpretations of the RPP in
which the concepts of \emph{seller} and \emph{client} find quite natural
translations as levels of a bilevel program. These matters are discussed in the
second chapter.

The introduction also contains a presentation of the RPP, and a section listing
various results from graph theory. Their interest lies in that they allow
translating statements about valid inequalities for some optimisation problem,
the Set Packing Problem (SPP), into statements about subgraphs of a certain
graph associated to the SPP.

\paragraph*{}

The second chapter presents four formulations for the RPP. Two bilevel
formulations are first introduced. In these, the seller is represented by the
first level problem, and each client is represented by one of the second level
problems. In this situation, the seller prices its products so as to maximise
his revenue, whereas clients choose a maximum-preference product of price less
than or equal to their budgets. Which variables to use gives rise to two
different formulations, \bnlp and $\rm{(BNL}^\textit{v}\rm{)}$. In
$\rm{(BNL}^\textit{p}\rm{)}$, the price of product $i$ is represented by $p_i$,
a real variable explicitly indicating $i$'s price.  Formulation \bnlv can be
obtained from \bnlp by employing a result from \cite{ru:nonparametric}. This
result states that in an optimal solution to the RPP, prices assigned to items
belong to the set of all client budgets.  Representing this set by $B$, the
price of item $i$ can then be represented by $v_i^\ell$. $v_i^\ell$ is a binary
variable indicating whether product $i$ is assigned the price $\ell \in B$.

Wishing to elude the high computational complexity of bilevel formulations, a
single-level formulation \bnl is derived from \bnlv by characterising solutions
to \bnlv's second level problem through linear restrictions. From (BNL), a
second single-level formulation \slnl is derived. This last formulation will
serve as basis for the rest of the work.

\paragraph*{}

The third chapter derives two linear formulations from \slnl. The reason for
this change is again the desire to develop a method of the least possible
computational complexity. Indeed, solving non linear problems is generally much
harder than solving linear ones. The problematic element in \slnl is its
objective function, which contains products of variables. Therefore,
linealisation of \slnl is performed by transforming its objective function into
a linear function. With this purpose, a new family of variables is defined which
allows expressing the objective function in a linear manner. Furthermore, it is
necessary to include several restrictions that give the new variables their
intended meaning.

The concrete election of which variables to add gives rise to two distinct
formulations: \slla and $\rm{(SLL}_2\rm{)}$. The new variables in \slla are
$z^k$, the benefit obtained by the seller from client $k$. New variables in
\sllb are more granular: $z^k_i$, representing the benefit obtained by the
seller by selling product $i$ to client $k$. Please note that if some client
$k_0$ does not purchase a particular product $i_0$, then $z_{i_0}^{k_0} = 0$.

Two families of valid inequations, one for each linear formulation, are defined.
These are of exponential size, so that it becomes necessary to study them
minutely in order to derive a separation algorithm for each family. These
algorithms are provided as entry a fractional solution for some linear
relaxation of the problem, and return a restriction from the family not
satisfied by the fractional solution. Such methods are useful for solving the
RPP by the branch and bound procedure.

\paragraph*{}

The objective of the fourth chapter, is to determine how good provided
formulations are. This is in stark contrast with chapters 2 and 3, which merely
tried to improve the existing formulations, without trying to assess how tight
their restrictions were. First, it is noted that several restrictions from
formulations \slla and \sllb give rise to the feasible region of a Set Packing
Problem (SPP). This finding allows us to employ results presented at the end of
Chapter \ref{chp:introduction} in order to determine which restrictions of \slla
and \sllb are facets of the corresponding set packing subproblems. In order to
do this, we need to consider the so-called intersection graph $G$ of the set
packing subproblem. An important result due to Padberg
\cite{pa:facial-structure} allows us to relate facets of the SPP to cliques in
$G$. This change of language from set packing to graph theory allows us to
conclude that most restrictions of the SPP are facets. This implies that no
trivial improvement can be made to formulations \slla and $\rm{(SLL}_2\rm{)}$.

\paragraph*{}

Finally, the fifth chapter presents the author's original contributions. These
are the extension of formulations \slla and \sllb to the envy-free CRPP and a
computational study of these formulations.

The Capacitated Rank Pricing Problem (CRPP) is an extension of the RPP. The
additions it introduces are that it permits defining stock limitations and that
it allows clients to have different budgets for different products. Stock
limitations allow modelling situations in which a certain product cannot be
produced without limitation. For instance, a theatre may offer 50 box seats and
200 seats in the stalls, but no more because of lack of space. CRPP has two
variants: with and without envy. A solution is said to present envy if some
customer $k$ wishes to acquire a product $i$ whose price is less than or equal
to his budget, but cannot do so because it has run out of stock. CRPP with envy
is the CRPP version allowing solutions to present envy. Conversely, the
envy-free version does not allow solutions that present envy. We have developed
two formulations to the envy-free CRPP based in formulations \slla and \sllb.

The fifth chapter also presents a computational study, in which the efficiency
of the proposed CRPP formulations is assessed. For this purpose, several CRPP
instances have been randomly generated.

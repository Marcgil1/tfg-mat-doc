\chapter{Conclusions}
\label{chp:conclusions}

In this work, we have studied the Rank Pricing Problem, an NP-hard integer
optimisation problem. In Chapter \ref{chp:problem-formulations}, a bilevel
formulation was first presented building on a natural interpretation of the
program as an interaction between seller and customer. Then, single-level
formulations were obtained from this bilevel one. In Chapter
\ref{chp:linealisation-and-strenghtening}, single-level linear formulations were
introduced, building on intuition from the formulations in Chapter
\ref{chp:problem-formulations}. Furthermore, families of valid inequalities of
exponential size were introduced, alongside a separation procedure for
dynamically introducing them into the formulation. Chapter \ref{chp:spp}
presented a theoretical study characterising the facet-defining restrictions of
(SPSP), a set packing subproblem found in formulations \slla and \sllb. The
results there introduced permitted assessing the tightness of \slla and \sllb's
restrictions.

Finally, Chapter \ref{chp:crpp-without-envy} contained the author's original
contributions. It presented the Capacitated Rank Pricing Problem, an extension
for the RPP allowing the seller to restrict the quantity of items available from
each product. Two formulations were developed for the envy free version of the
problem based on \slla and $\rm{(SLL}_2\rm{)}$. This complements the study
carried out in \cite{do:envy}, which gave formulation for the more complex case
of CRPP with envy. Finally, Chapter \ref{chp:crpp-without-envy} also presented a
computational study on the proposed formulations for the CRPP, thus assesssing
their performance.

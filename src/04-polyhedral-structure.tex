\chapter{Polyhedral structure of the set packing subproblem}
\label{chp:spp}

\section{Introduction} % -------------------------------------------------------
\label{sec:spp:introduction}

All of the work presented up to this point has been concerned with the
development of an adequate formulation for the RPP. However, despite all of this
work, no mention has been made to precisely how good the obtained formulations
are. However, some precise statements can be made regarding the tightness of
some restrictions in formulations \slla and \sllb, their presentation being this
chapter's objective. This will be done noting that a particular set of
restrictions of these formulations corresponds to a Set Packing Problem (see
Definition \ref{def:spp}), thus being able to apply to it Padberg's
characterisation of SPP facets through intersection graphs. For a review of
these topics, please refer to Section \ref{sec:int:graph-theory-and-spp} in the
Introduction.

Consider the families of restrictions \eqref{eq:sll1:b} - \eqref{eq:sll1:e} of
problem \slla, which we will refer to as \emph{set packing subproblem} (SPSP):

{
    \newcommand{\bover} {\overline{B(k, i)}}
    \newcommand{\bne}   {\bover\neq\varnothing}
    \newcommand{\sumi}  {\sum_{\is}}
    \newcommand{\sumlm} {\sum_{\ell=1}^{M}}
    \newcommand{\sumj}  {\sum_{j\in\bover}}
    \newcommand{\sumls} {\sum_{\ell=1}^{\sigma(k)}}
    \newcommand{\sumlsm}{\sum_{\ell=\sigma(k)+1}^{M}}
    
    \begin{eqnarray}
        \rm{(SPSP)}
            &    & \sumi  x_i^k                \leq 1, \quad\forall\kk
                                                            \label{eqn:spsp:a}\\
            &    & \sumlm v_i^\ell             \leq 1, \quad\forall\ii
                                                            \label{eqn:spsp:b}\\
            &    & \sumj x_j^k+\sumls v_i^\ell \leq 1, \quad\forall\kk,\is:\bne
                                                            \label{eqn:spsp:c}\\
            &    & x_i^k + \sumlsm v_i^\ell    \leq 1, \quad\forall\kk,\is.
                                                            \label{eqn:spsp:d}
    \end{eqnarray}
}

\noindent
This terminology is justified by noting that the coefficient matrix of all these
restrictions has binary entries, and that the right-hand side vector's
components are all ones, thus corresponding this set of restrictions to the
feasible region of a Set Packing Problem. Please note that these very
restrictions can also be found in Formulation \sllb, in constraints
\eqref{eq:sll2:b} - \eqref{eq:sll2:e}. Thus, by studying (SPSP) we will be able
to derive conclusions equally applicable to both of these formulations.

However, due to technical details, we will have to add some restrictions to
(SPSP). These additions were implicitly assumed in \cite{ca:rpp}, but here, we
will state them explicitly. We will namely add all valid restrictions of the form
$x_i^k + x_j^{k'} \leq 1$. An explicit enumeration of these will be obtained in
Point 3 of Proposition \ref{pro:characterisation}.

Let us now represent by $G = (V, E)$ the intersection graph corresponding to
(SPSP), as described in Definition \ref{def:intersection-graph}, and by $P(G)$ the convex
hull of all feasible solutions of (SPSP). Since tightness of restrictions has a
direct influence in the efficiency of resolution, quality of formulations \slla
and \sllb may be checked by ensuring that many of the inequalities
\eqref{eqn:spsp:a} - \eqref{eqn:spsp:d} are facets of $P(G)$. This will be
obtained as conclusion to the main theorems presented in Section
\ref{sec:spp:facets}.  Introduction of the intersection graph $G$ does not serve
merely explanatory purposes: it stands at the hearth of theorems
\ref{thm:xineqs}, \ref{thm:xfacets} and \ref{thm:vineqs}. The point in the
proofs of these theorems in which $G$ will allow us to make statements about
facets is a notable theorem due to Padberg \cite{pa:facial-structure}. It is
therefore necessary to give a precise description of $G$, which will be done in
next section.

\section{Description of the intersection graph} % ------------------------------
\label{sec:spp:intersection-graph}

In order to handle the intersection graph $G$, a characterisation of it through
variables $x$ and $v$ must be obtained, since phrasing any argument in terms of
(SPSP)'s constraint matrix and its rows would be overly cumbersome. However, as
will be seen in the proof, these characterisations are almost direct
translations of the constraint matrix's structure.

\begin{proposition}
    \label{pro:characterisation}

    If $G$ is the intersection graph associated to \rm{(SPSP)}, then the
    following statements on $G$'s nodes hold:
    \begin{enumerate}
        \item % 1
	    For any client \kk and distinct products $i \neq j$ in $S^k$, nodes
	    $x_i^k$ and $x_j^k$ are adjacent.
	\item % 2
	    For any pair of distinct clients $k, k' \in K$, $k \neq k'$ and
	    product \ii, nodes $x_i^k$ and $x_j^{k'}$ are \emph{not} adjacent.
	\item % 3
	    For any pairs $k \neq k'$ and $i \neq j$ of distinct clients and
	    products, adjacency of nodes $x_i^k$ and $x_j^{k'}$ is equivalent to
	    conditions $\sigma(k) \geq \sigma(k')$ and $j \in B(k, i)$.
	\item % 4
	    For any client \kk, product \ii , and price index \lm, adjacency of
	    nodes $x_i^k$ and $v_i^\ell$ is equivalent to the condition $\ell >
	    \sigma(k)$.
	\item % 5
	    For any client \kk, distinct products $i \neq j$ and price index
	    \lm, adjacency of nodes $x_i^k$ and $v_j^\ell$ is equivalent to
	    conditions $\ell \leq \sigma(k)$ and $j \in B(k, i)$.
	\item % 6
	    For any product \ii, and distinct price indices $\ell \neq \ell'$,
	    nodes $v_i^\ell$ and $v_i^{\ell'}$ are adjacent.
	\item % 7
	    For any distinct products $i \neq j$ and pair of price indices
	    $\ell, \ell' \in \{1, \ldots, M\}$, nodes $v_i^\ell$ and
	    $v_j^{\ell'}$ are \emph{not} adjacent.
    \end{enumerate}
\end{proposition}

\begin{proof}
    This result will be proved by relating each of the assertions with their
    translation into statements about (SPSP)'s constraint matrix, which we will
    henceforth refer to as $A$. Please recall that, from the definition of
    intersection graph (Definition \ref{def:intersection-graph}), two variables
    are adjacent iff a row of $A$ exists which simultaneously assigns them a
    coefficient of 1. We will separately prove each statement:
    \begin{enumerate}
        \item % 1
	    For any client $\kk$ and distinct products $i, j \in S^k$, variables
	    $x_i^k$ and $x_j^k$ appear in the left-hand side of the
	    \eqref{eqn:spsp:a} restriction associated to $k$. Therefore, $x_i^k$
	    and $x_j^k$ are adjacent in this case.
        \item % 2
	    In this case, variables $x_i^k$ and $x_i^{k'}$ do not simultaneously
	    appear on the left-hand side of any of (SPSP)'s restrictions. This
	    corresponds to the fact that knowing that $k$ purchases a product
	    $i$ does not give any clue on whether $k'$ will buy $i$.
        \item % 3
	    If $x_i^k = 1$, that is, if $k$ purchases product $i$, then $k$
	    cannot afford any product $k$-better than $i$, since otherwise he
	    would have bought it instead of $i$. Similarly, no client $k'$ with
	    a tighter budget, $\sigma(k') \leq \sigma(k)$, will be able to
	    afford it either. Therefore, $x_j^{k'} = 1$, as stated on Point 3.
	    Conversely, knowing that $k$ does not purchase $i$ provides us with
	    no information on whether clients $k'$ richer than $i$ will be able
	    to afford products $\overline{B(k, i)} \cup \{i\}$.
        \item % 4
	    The only way $x_i^k$ and $v_i^\ell$ can appear in the same left-hand
	    side is in inequality family \eqref{eqn:spsp:d}. This happens
	    precisely when $\ell \geq \sigma(k) + 1$.
        \item % 5
	    The only way $x_i^k$ and $v_j^\ell$ can appear in the same
	    constraint is in constraint family \ref{eqn:spsp:c}, and in this
	    family this only occurs for $l \leq \sigma(k)$ and $j \in B(k, i)
	    \iff i \in \overline{B(k, j)}$.
        \item % 6
	    It suffices to note that, for any \ii, all the $v_i^\ell$ are added
	    together in restriction family \eqref{eqn:spsp:b}.
        \item % 7
	    In none of the (SPSP) restrictions two $v$ variables with different
	    $i$ indices get added together. Thus, $v_i^\ell$ and $v_j^{\ell'}$
	    are never adjacent if $i \neq j$.
    \end{enumerate}
\end{proof}

\section{Description of SPP facets} % ------------------------------------------
\label{sec:spp:facets}

Having introduced last section's result on the structure of the intersection
graph's structure we may now head on to present this chapter's main results. In
order to do so, however, we will need to introduce some further nomenclature
extending the definition of the $B(k, i)$ sets:

\begin{definition}
    \label{def:better-set}

    If \kk is a customer and $P \subseteq S^k$, then we define
    $
        B(k, P) = \{ i \in S^k \mid i >_k j\ \forall j \in P\}
    $.
    That is, $B(k, P)$ is the set of products which $k$ prefers over all
    products in $P$. Similarly, we may define $\overline{B(k, P)}$ to be the set
    of products which are less preferred by $k$ than any product in $P$. More
    precisely,
    $
        \overline{B(k, P)} = \{ i \in S^k \mid i <_k j\ \forall j \in P\}
    $.
    Please note that $\overline{B(k, P)}$ need not be $B(k, P)$'s complement.
    For instance, products in $P$ do not belong to any of these sets.
\end{definition}

Furthermore, an auxiliary result must be presented. It is a natural result
stating that cliques containing some pair $v_i^{\ell_1}$ and $v_i^{\ell_2}$ of
$v$-variables also contains all $v$-variables with $\ell$ between $\ell_1$ and
$\ell_2$:

\begin{lemma}
    \label{lem:price-index-interval}

    Any clique in $G$ containing nodes $v_i^{\ell_1}$ and $v_i^{\ell_2}$ for
    some product $i$ and price indices $\ell_1 < \ell_2$ also contains all nodes
    $v_i^\ell$ with $\ell_1 < \ell < \ell_2$.
\end{lemma}

\begin{proof}
    Given a clique $G'$ in $G$ containing $v_i^{\ell_1}$ and $v_i^{\ell_2}$, we
    will prove that if $\ell_1 < \ell < \ell_2$, then $v_i^\ell$ is adjacent to
    every node in $G'$. Since $G$ contains $x$-nodes and $v$-nodes, we will
    prove that $v_i^\ell$ is adjacent to nodes in each of these families.

    Firstly, if $v_j^{\ell'} \in G'$, then it is adjacent to $v_i^{\ell_1}$,
    since $G'$ is a complete subgraph. By parts (6) and (7) of Proposition
    \ref{pro:characterisation} this means that $i = j$. By these very results,
    we then get that $v_j^{\ell'}$ is adjacent
    to $v_i^\ell$.

    Secondly, if $x_i^k \in G'$, then it is adjacent to $v_i^{\ell_1}$, since
    $G'$ is a complete subgraph. Thus, by Proposition \ref{pro:characterisation}
    part (4), it must be the case that $\sigma(k) < \ell_1$. This very result
    asserts then that $\sigma(k) < \ell_1 < \ell$ implies $x_i^k$ is adjacent to
    $v_i^\ell$.

    Lastly, if $x_j^k \in G'$ for some $j \neq i$, then it is adjacent to
    $v_i^{\ell_2}$, since $G'$ is a complete subgraph. Thus, by Proposition
    \ref{pro:characterisation} part 5, $j \in B(k, i) \iff i \in \overline{B(k,
    j)}$ and $\sigma(k) \geq \ell_2$. Therefore, since $\ell < \ell_2 \leq
    \sigma(k)$, that same result implies that $x_j^k$ is adjacent to $v_i^\ell$,
    thus ending the proof.
\end{proof}

It is now possible for us to introduce the main results in this section. These
are \ref{thm:xineqs} and \ref{thm:xfacets}, relating to $x$-variables and
\ref{thm:vineqs} relating to $v$-variables. For each set of variables, they
present valid families of restrictions and conditions over these families for
their restrictions to be facet-defining. The importance of these four results
will become clear at the end of the section, when we discuss their implications
for formulations \slla and \sllb.

\begin{theorem}
    \label{thm:xineqs}

    For any set of customers $\{k_2, \ldots, k_n\}$ with $n \geq 2$ and ordered
    non-decreasingly by purchase power, $\sigma(k_2) \leq \cdots \leq
    \sigma(k_n)$, and for any non empty and pairwise disjoint sets $P^{k_q}
    \subseteq S^{k_q}$, $q \in \{2, \ldots, n\}$ such that
    \begin{equation*}
        P^{k_q} \subseteq
                \left(
                    \bigcap_{r=2:\\ \sigma(k_r) < \sigma(k_q)}^{q-1}
                        \overline{B(k_q, P^{k_r})}
                \right)
            \bigcap
                \left(
                    \bigcap_{r=2:\\ \sigma(k_r) = \sigma(k_q)}^{q-1}
                        \left(
                            \overline{B(k_q, P^{k_r})} \cup B(k_r, P^{k_r})
                        \right)
                \right)
    \end{equation*}
    \noindent
    for all $q \in \{3, \ldots, n\}$, the following are valid restrictions for
    formulations \slla and \sllb:
    \begin{equation}
        \label{eqn:xineqs}
        \sum_{q=2}^n \sum_{j \in P^{k_q}} x_j^{k_q} \leq 1.
    \end{equation}
\end{theorem}

\begin{proof}
    Let us denote by $G = (V, E)$ the intersection graph of the set packing
    subproblem, and let $Q = (V', E')$ be a clique of $G$ containing only $x$
    variables. Choose $k_2$, a client with minimum budget in the clique, and
    $P^{k_2} \subseteq S^{k_2}$ such that $x_j^{k_2} \in V'$ for all $j \in
    P^{k_2}$.

    Let us consider other customers $k_q$, $q = 3, \ldots, n$ ordered such that
    $\sigma(k_2) \leq \cdots \leq \sigma(k_n)$, and alongside nonempty sets of
    products $P^{k_q} \subseteq S^{k_q}$ such that $x_j^{k_q} \in V'$, $\forall
    j \in P^{k_q}, q = 3, \ldots, n$. Then, by the second point of Proposition
    \ref{pro:characterisation}, the $P^{k_2}, \ldots, P^{k_n}$ are pairwise
    disjoint. Furthermore, they verify the following two inclusions:
    \begin{enumerate}
        \item % 1
            \[
                P^{k_q} \subseteq
                    \bigcap_{r=2 \mid \sigma(k_r) < \sigma(k_q)}^{q-1}
                        \overline{B(k_q, P^{k_r})}
                , \quad \forall q = 3, \ldots, n
            \]
	    If this relation didn't hold, there would exist some $k_r$ with
	    $\sigma(k_r) < \sigma(k_q)$ and products $i \in P^{k_r}$ and $j \in
	    P^{k_q}$ with $x_i^{k_r}, x_j^{k_q} \in V'$ and $j \notin
	    \overline{B(k_q, i)}$. However, this is absurd, because the third
	    point of Proposition \ref{pro:characterisation} would then imply
	    that $x_i^{k_r}$ and $x_j^{k_q}$ would not be neighbours in $G$,
	    contradicting that $V'$ is a complete subgraph of $G$.
        \item % 2
            \[
                P^{k_q} \subseteq
                    \bigcap_{r=2 \mid \sigma(k_r) < \sigma(k_q)}^{q-1}
                        \left(
                                \overline{B(k_q, P^{k_r})}
                            \cup
                                B(k_r, P^{k_r})
                        \right)
                , \quad \forall q = 3, \ldots, n
            \]
	    If this were not the case, there would exist some $k_r$ with
	    $\sigma(k_r) = \sigma(k_q)$ and products $i \in P^{k_r}$ and $j \in
	    P^{k_q}$ such that $x_i^{k_r}, x_j^{k_q} \in V'$. However, the third
	    point of Proposition \ref{pro:characterisation} implies that
	    $x_i^{k_r}$ and $x_j^{k_j}$ are not neighbours in $G$. As before,
	    this is absurd, since it contradicts that $V$ is a complete graph.
    \end{enumerate}

    In summary, these restrictions ensure that the $x$ variables appearing in a
    sum of the type \ref{eqn:xineqs} give rise to a complete subgraph $V'$,
    which implies that inequalities of family \ref{eqn:xineqs} are valid
    constraints for the set packing subproblem.
\end{proof}

\begin{theorem}
    \label{thm:xfacets}
    Inequalities of the family \eqref{eqn:xineqs} are facets of \rm{(SPSP)} iff
    \begin{enumerate}
        \item
            There does not exist any client $k_0 \in K$ and client $i_0 \in S^k$
            satisfying the following conditions:
            \begin{enumerate}
        	\item % 1
        	    $i_0$ belongs to $B(k_0, P^{k_0})$ for all $q \in \{2, \ldots, n\}$
        	    such that $\sigma(k_q) \geq \sigma(k_0)$.
        	\item % 2
        	    $i_0$ belongs to $\overline{B(k_0, P^{k_0})}$ for all $q \in \{2,
        	    \ldots, n\}$ such that $\sigma(k_q) \leq \sigma(k_0)$.
            \end{enumerate}
       \item
	    \(
                    \left|
                        \bigcap_{q=2:\sigma(k_q) = \sigma(k_2)}^n
                            P^k_q
                    \right|
	        \geq
	            2
	    \)
    \end{enumerate}
    \noindent
    Finally, all facets of \rm{(SPSP)} whose associated clique in $G$ consist of
    only $x$-variables are contained in family \eqref{eqn:xineqs}.
\end{theorem}

\begin{proof}
    Let us suppose that conditions above are not fulfilled, and define $(V, E)$,
    $(V', E')$ as in the proof above. We have two options:
    \begin{enumerate}
        \item
	    There exists some $(k_0, i_0) \in K \times S^{k_0}$ fulfilling the
	    above conditions over $(k_0, i_0)$. Then, Point 3 of Proposition
	    \ref{pro:characterisation} we have that the node $x_{i_0}^{k_0}$ is adjacent
	    to every node in $V'$, contradicting the fact that $V'$ is a clique.
	\item
	    Last condition,
            \(
                    \left|
                        \bigcap_{q=2:\sigma(k_q) = \sigma(k_2)}^n
                            P^k_q
                    \right|
	        \geq
	            2
	    \),
	    holds. Then, there cannot exist any $v$-variable that is adjacent to
	    all other nodes within subgraph $V'$. If this were not the case, we
	    would have $P^{k_2}=\{i\}$, and one of $\sigma(k_2) < \sigma(k_3)$
	    or $n=2$ would hold. However, this would imply that
	    $v_i^{\sigma(k_2)+1}$ be adjacent to every node in $V'$,
	    contradicting the fact that the subgraph be maximal.
    \end{enumerate}
    Otherwise, the construction in last proof above ensures that the graph be
    maximal, and thus, that the corresponding inequality is a facet.
\end{proof}

\begin{theorem}
    \label{thm:vineqs}

    For any nonempty set of price indices
    $
        L =         \{\ell_1, \ldots, \ell_p\}
          \subseteq \{1,      \ldots, M     \}
    $,
    product \ii and
    \begin{itemize}
        \item % \ell_1 > 1
	    if $\ell_1 > 1$, customer $k_1$ such that $\sigma(k_1) = \ell_1 -
	    1$, $i \in S^{k_1}$, and set $P^{k_1} = \{i\}$; otherwise $P^{k_1} =
	    \varnothing$;
	\item % \ell_p < M
	    if $\ell_p < M$, customers $k_2, \ldots, k_n$, $n \geq 2$, such that
	    $
	        \ell_p = \sigma(k_2) \leq \ldots \leq \sigma(k_n)
	    $
	    ($n = 1$ otherwise) and non empty pairwise disjoint sets of products
	    $P^{k_q} \subseteq S^{k_q} - \{i\}$, $q \in \{2, \ldots, n\}$ such
	    that $P^{k_2} \subseteq \overline{B(k_2, i)}$ and
            \begin{equation*}
                P^{k_q} \subseteq
                        \left(
                            \bigcap_{r=2:\\ \sigma(k_r) < \sigma(k_q)}^{q-1}
                                \overline{B(k_q, P^{k_r})}
                        \right)
                    \bigcap
                        \left(
                            \bigcap_{r=2:\\ \sigma(k_r) = \sigma(k_q)}^{q-1}
                                \left(
                                        \overline{B(k_q, P^{k_r})}
                                    \cup
                                        B(k_r, P^{k_r})
                                \right)
                        \right)
            \end{equation*}
	    for all $q \in \{3, \ldots, n\}$, the following are valid
	    restrictions for formulations \slla and \sllb:
	    \begin{equation}
    	        \label{eqn:vineqs}
    	            \sum_{\ell \in L} v_i^\ell
    	        +
    	            \sum_{q=1}^n \sum_{j \in P^{k_q}} x_j^{k_q}
    	        \leq
    	            1
	    \end{equation}
    \end{itemize}
    
    In addition, inequalities of family \eqref{eqn:vineqs} are facets of
    \rm{(SPSP)} iff there does not exist any client $k_0 \in K$ and $i_0 \in
    S^{k_0} - \{i\}$ satisfying the following conditions:
    \begin{itemize}
        \item % 1
            $\sigma(k_0) \geq \ell_p$
        \item % 2
	    $i_0$ belongs to $B(k_q, P^{k_q})$ for all $q \in \{1, \ldots, n\}$
	    such that $\sigma(k_q) \geq \sigma(k_0)$.
        \item % 3
	    $i_0$ belongs to $\overline{B(k_0, P^{k_q})}$ for all $q \in \{1,
	    \ldots, n\}$ such that $\sigma(k_q) \leq \sigma(k_0)$.
    \end{itemize}
    \noindent
    Finally, all facets of \rm{(SPSP)} whose associated clique in $G$ consists
    of only $x$-variables are contained in family \eqref{eqn:vineqs}.
\end{theorem}

\begin{proof}
    The proof will be similar to that of Theorem \ref{thm:xineqs}. However, the
    special structure for $v$ variables suggested by Lemma
    \ref{lem:price-index-interval} will induce a case analysis on the
    superindices of $v_i^\ell$.

    Let us denote by $G = (V, E)$ the intersection graph of the set packing
    subproblem (SPSP), and let $Q = (V', E')$ be a clique of $G$ containing some
    $v$ variables. Since all variables within the clique are adjacent, Point 7
    of Proposition \ref{pro:characterisation}, all variables within clique $V'$
    must have the same subindex, which we may represent as $i$. Similarly, Lemma
    \ref{lem:price-index-interval} asserts that all variables have consecutive
    superindices, their set being represented as $L = \{\ell_1, \ldots,
    \ell_p\}$. We will now prove that the conditions in the statement are all
    met by following a case analysis on the extent of $L$.

    \begin{itemize}
        \item
            $L = \{1, \ldots, M\}$.
            
	    If this is the case, we may employ Proposition
	    \ref{pro:characterisation} in order to conclude that any $x$ node
	    $x_j^k$ in the neighbourhood of $L$'s associated $v$ variables,
	    $v_i^1, \ldots, v_i^M$ must satisfy conditions $\sigma(k) = M$ and
	    $j \in \overline{B(k, i)}$. Since $\sigma(k) = M$ means $k$ is a
	    richest customer, and the richest customers can purchase the product
	    they prefer the most, we can infer that $\varnothing = P^{k_2} =
	    \cdots = P^{k_n}$. Furthermore, Point 4 of Proposition
	    \ref{pro:characterisation} does not provide us with any node which
	    is adjacent to all $v$-nodes $v_i^\ell$, for all $\ell$. We may
	    therefore conclude that $\{v_i^\ell \mid \ell \in L\}$ is a maximal
	    complete subgraph of the intersection graph $G$.
	\item
	    There is some $\ell_1 > 1$ such that $L = \{\ell_1, \ldots, M\}$.

	    In this case, there exists a node in the clique adjacent to all
	    $v_i^\ell$ with $\ell \geq \ell_1$, but not to $v_i^{\ell_1-1}$.
	    Furthermore, Proposition \ref{pro:characterisation} and Lemma
	    \ref{lem:price-index-interval} allow us to more explicitly classify
	    this node as an $x$-variable. Let us consider this node is $x_j^k$
	    for some client $k$ and product $j$. By applying Point 5 of
	    Proposition \ref{pro:characterisation} we cannot obtain any other
	    node adjacent to $v_i^M$, so it follows that $\varnothing = P^{k_2}
	    = \cdots = P^{k_n}$. Then, Point 4 of the same proposition asserts
	    that $x_j^k$ has to be adjacent to all $v_i^\ell$ with $\ell \geq
	    \ell_1$.. Therefore, $j = i$, and there exists some customer $k_1$
	    with $\sigma(k_1) < \ell_1$ and $P^{k_1} = \{i\}$ such that $k =
	    k_1$. Since, by construction, $x_i^{k_1}$ is not adjacent to $v$-node
	    $v_i^{\ell_1-1}$, Point 4 allows us to conclude that $\sigma(k_1)
	    \geq \ell_1 - 1$, and therefore $\sigma(k_1) = \ell_1 - 1$. Thus,
	    the first point in the statement is verified.

	    Let us now suppose that there exists some other node $x_j^k \in V'$.
	    In this case, Point 4 of Proposition \ref{pro:characterisation}
	    asserts that $x_j^k$ is adjacent to all $v_i^\ell$ with $\ell \geq
	    \ell_1$. Consequently, $j$ must equal $i$. However, in this case
	    Point 2 of the same proposition asserts that $x_i^k$ and $x_i^{k_1}$
	    are not adjacent for any $k \neq k_1$, and therefore $\{v_i^\ell
	    \mid \ell \geq \ell_1\}\cup\{x_i^{k_1}\}$ is in this case a clique
	    in the intersection graph $V'$.
	\item
	    There is some $\ell_p < M$ such that $L = \{1, \ldots, \ell_q\}$.

	    In a similar manner to that of last case, relations $v_i^\ell \notin
	    V'$ for $\ell > \ell_p$, Lemma \ref{lem:price-index-interval} and
	    Proposition \ref{pro:characterisation} imply the existence of an
	    $x$-variable $x_{i_0}^k$ in $V'$ adjacent to $v_i^{\ell_p}$ but not
	    to $v_i^{\ell_p+1}$. We cannot obtain any node adjacent to $v_i^1$
	    by means of Point 4 of Proposition \ref{pro:characterisation}. Thus,
	    $P^{k_1} = \varnothing$, and, by Point 5, $x_{i_0}^k$ and $v_i^\ell$
	    are adjacent for all $\ell \leq \ell_p$. We may thus infer the
	    existence of a customer $k_2$ with budget $\sigma(k_2) \geq \ell_p$
	    and a subset of products $P^{k_2} \subseteq \overline{B(k_2, i)}$
	    such that $i_0 \in P^{k_2}$and, by Point 5, $x_j^{k_2} \in V'$ for
	    all $j \in P^{k_2}$. Furthermore, that $x_{i_0}^{k_2}$ is not
	    adjacent to $v$-node $v_i^{\ell_p+1}$ allows us to conclude relation
	    $\sigma(k_2) = \ell_p$ in the theorem statement. If there also exist
	    customers $k_q$ for $3 \leq q \leq n$ with $\sigma(k_2) \leq
	    \sigma(k_3) \leq \cdots \leq \sigma(k_n)$, and, associated to them,
	    nonempty subsets $P^{k_q} \subseteq S^{k_q}$ such that $x_j^{k_q}
	    \in F'$ for all $j \in P^{k_q}$, then an application of Point 2
	    allows us to conclude that the $P^{k_i}$, $1 \leq i \leq n$ are
	    pairwise disjoint. Furthermore, none of the $P^{k_q}$, $3 \leq q
	    \leq n$, contains product $i$. If this were not the case, we would
	    have $x_i^{k_q} \in V'$ for some $k_q$ with $\sigma(k_q) \geq
	    \ell_p$ not adjacent to $v_i^{\ell_p}$, contradicting the assumption
	    of $V'$ being a complete graph.
	\item
	    There exist $\ell_1 > 1$ and $\ell_p < M$ such that $L = \{\ell_1,
	    \ldots, \ell_p\}$.

	    This case is handled by combining the arguments of the two previous
	    cases.
    \end{itemize}
\end{proof}

Provided with these results, we are now able to assess the quality of (SPSP)'s
inequalities. Indeed, cases 1 and 2 from the proof of Theorem \ref{thm:vineqs}
assert that restrictions from families \eqref{eqn:spsp:b} and \eqref{eqn:spsp:d}
are always facet-defining. Furthermore, theorem \ref{thm:xfacets} allows us to
assert that all inequalities from \eqref{eqn:spsp:a} for which $|S^k| \geq 2$
and there does not exist any pair $(k_0, i_0)$ in $K \times S^{k_0}$ satisfying
$\sigma(k_0) \geq \sigma(k)$ and $i_0 \in \overline{B(k_0, S^k)}$. Certainly,
this does not allow us to conclude that all inequalities in \eqref{eqn:spsp:a}
are facet-defining, but it indicates that most of them are. In a similar manner,
conditions from Theorem \ref{thm:vineqs} allow us to conclude that most
restrictions of family \eqref{eqn:spsp:c} are facet-defining.

It should be noted that being facet-defining in (SPSP) does not imply being
facet-defining in either \slla or \sllb, for these formulations also include $z$
variables and further restrictions. However, it does indicate that they are very
tight, thus leading to fast solving times.

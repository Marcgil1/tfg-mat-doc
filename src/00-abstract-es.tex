\selectlanguage{spanish}
\pagestyle{plain}

\chapter*{Resumen (Español)} % -------------------------------------------------

El objeto del presente trabajo es el Problema de Asignación de Precios con Rango
(en inglés, \emph{Rank Pricing Problem}, RPP). Éste es un problema de
optimización consistente en determinar de qué forma debe un productor asignar
precios a sus productos de manera que se maximice su beneficio. A continuación,
presentaremos el RPP y daremos al lector un resumen de la estructura del
trabajo.

\paragraph*{}

El Problema de Asignación de Precios con Rango supone la existencia de un
vendedor y de un conjunto de clientes. El vendedor ofrece diversos productos, y
desea asignar un precio a cada uno de manera que se maximice su beneficio. A su
vez, cada cliente viene caracterizado por una lista de preferencias. Esta lista
está conformada por un conjunto de productos de entre los ofertados por el
vendedor. Los productos de estas listas de preferencias están totalmente
ordenados, de manera que, para cada par de productos $i_1$ e $i_2$ deseados por
un cliente $k$, bien $i_1$ es estrictamente mejor que $i_2$ para $k$, bien $i_2$
es estrictamente mejor que $i_1$ para $k$. Todo cliente tiene además un
presupuesto, que es el máximo gasto que se puede permitir. Así pues, cada
cliente desea adquirir exactamente un producto, que sea de máxima preferencia
para él, pero cuyo precio sea menor o igual que su presupuesto.

El RPP ha sido tratado en profundidad en el artículo \cite{ca:rpp}. Es de esta
fuente de donde se han extraído la mayor parte de resultados de los capítulos
segundo, tercero y cuarto. A menos que se indique lo contrario, todo resultado
se debe entender contenido en el artículo de Calvete y otros.

Este trabajo se encuentra dividido en seis capítulos, cuyos contenidos se
resumen a continuación.

\paragraph*{}

El desarrollo del trabajo requiere de resultados y definiciones pertenecientes a
tres áreas diferentes de la optimización: programación lineal, programación
entera y programación binivel. Éstos se presentan en el primer capítulo.

Un entendimiento de programación lineal es requisito necesario para comprender
los modelos de programación entera que forman el núcleo de este trabajo. Por
este motivo se le dedica la primera sección de la introducción. Tras una breve
reseña histórica de este paradigma, se dan varias de sus definiciones
elementales. Además, se presentan brevemente varios resultados referidos a
teoría de la dualidad. Éstos son usados en el segundo capítulo, donde permiten
derivar una formulación de un solo nivel a partir de las formulaciones binivel
inicialmente definidas.

La programación entera abarca la segunda sección de la introducción. Este
paradigma permite definir la mayoría de las formulaciones del trabajo. Se
presentan pues, varias definiciones elementales, y se analiza su complejidad
computacional y dos métodos de resolución: ramificación y acotación, y
ramificación y corte. Además, se introducen varios resultados relacionados con
matrices totalmente unimodulares que permiten resolver algunos problemas enteros
como si fueran lineales. Estos resultados serán de interés en el segundo
capítulo.

Finalmente, la programación binivel abarca la tercera sección de la
introducción. Esta rama de la optimización estudia la definición de problemas
con una estructura en dos niveles. En el problema de primer nivel, se desea
resolver un problema de programación lineal o entera. Además, entre las
restricciones de este problema de primer nivel se requiere que algunas variables
sean solución de un problema de segundo nivel. La programación binivel permite
dar en el segundo capítulo dos formulaciones del Problema de Asignación de
Precios con una interpretación muy natural de los conceptos de cliente y
vendedor.

La introducción contiene, además, una introducción al RPP, y una sección
presentando varios resultados de teoría de grafos. Estos resultados serán la
base del cuarto capítulo. Su interés se debe a que proporcionan una manera de
traducir afirmaciones sobre desigualdades válidas para un problema de
optimización, el problema de empaquetamiento de conjuntos (SPP), en afirmaciones
sobre subgrafos de un cierto grafo asociado al SPP.

\paragraph*{}


El segundo capítulo presenta cuatro formulaciones para el RPP. En primer lugar
se presentan dos formulaciones binivel en que se asigna al vendedor un problema
de primer nivel, y a cada uno de los clientes un problema de segundo nivel. Así
pues, el vendedor elige los precios de los productos maximizando su beneficio,
mientras que los clientes buscan elegir un producto dentro de su presupuesto con
\emph{nivel de preferencia} máximo. La elección concreta de variables da lugar a
dos formulaciones, $\rm{(BNL}^{\textit{p}}\rm{)}$ y
$\rm{(BNL}^{\textit{v}}\rm{)}$.  $\rm{(BNL}^{\textit{p}}\rm{)}$ nace de
representar el precio del producto $i$ mediante la variable $p_i$, que es una
variable real indicando el precio de $i$. La formulación
$\rm{(BNL}^{\textit{v}}\rm{)}$ se obtiene a partir de
$\rm{(BNL}^{\textit{p}}\rm{)}$ notando la aplicabilidad de un resultado dado en
\cite{ru:nonparametric}. Éste asegura que en la solución óptima del RPP los
precios asignados pertenecen al conjunto de diferentes presupuestos de los
clientes. Definiendo $B$ como el conjunto de diferentes presupuestos, podemos
entonces representar el precio del producto $i$ mediante $v_i^\ell$, que es una
variable binaria indicando si el producto $i$ recibe el precio $\ell \in B$.

Deseando eludir la alta complejidad computacional de las formulaciones binivel,
se deriva de $\rm{(BNL}^{\textit{v}}\rm{)}$ una formulación de un solo nivel,
(BNL), fruto de caracterizar las soluciones del problema de segundo nivel
mediante varias restricciones lineales. A partir de (BNL) se obtiene una segunda
formulación de un solo nivel (SLNL), que servirá de base al resto del trabajo.

\paragraph*{}

El tercer capítulo parte de la formulación (SLNL) para obtener dos formulaciones
lineales. El motivo de este cambio vuelve a ser el deseo de resolver el RPP con
un método de la menor complejidad computacional posible pues, efectivamente, la
resolución de los problemas no lineales es notoriamente más compleja que la de
los lineales. El elemento problemático en (SLNL) es su función objetivo, que
contiene productos de variables. Así pues, la linealización de esta formulación
se realiza transformando la función objetivo en una función lineal. Se define
para ello una nueva familia de variables que permiten expresar la función
objetivo de manera lineal. Además, es necesario incluir varias restricciones que
permiten dar sentido a las nuevas variables.

La elección concreta de nuevas variables da lugar a dos formulaciones lineales
distintas: $\rm{(SLL}_1\rm{)}$ y $\rm{(SLL}_2\rm{)}$. Las nuevas variables de la
primera son $z^k$, el beneficio obtenido por el vendedor del cliente $k$. Las de
la segunda son más granulares, $z^k_i$, representando el beneficio obtenido por
el vendedor de vender el producto $i$ al cliente $k$. Nótese que si en la
solución óptima un cliente $k_0$ no adquiere un producto $i_0$, entonces
$z_{i_0}^{k_0} = 0$.

También con ánimo de mejorar el tiempo de resolución se definen dos familias de
desigualdades válidas, una para cada formulación lineal. Éstas resultan tener un
tamaño exponencial, por lo que se hace necesario estudiarlas minuciosamente y
dar un algoritmo de separación para cada familia. Estos algoritmos toman como
entrada la solución fraccionaria de alguna relajación lineal del problema y
devuelven una restricción de la familia que no sea cumplida por la solución
fraccional. Tal procedimiento resulta útil para resolver el RPP mediante
ramificación y corte.

\paragraph*{}

A diferencia de los dos anteriores, el objetivo del cuarto capítulo no es
mejorar las formulaciones obtenidas, sino determinar cuán buenas son. Se
comienza notando que varias de las restricciones de las formulaciones
$\rm{(SLL}_1\rm{)}$ y $\rm{(SLL}_2\rm{)}$ forman lo que se conoce como un
Problema de Empaquetamiento de Conjuntos (en inglés, \emph{Set Packing Problem},
SPP). Tal descubrimiento nos permite emplear las técnicas y resultados
presentados al término del Capítulo \ref{chp:introduction} para determinar
cuáles de las restricciones de $\rm{(SLL}_1\rm{)}$ y $\rm{(SLL}_2\rm{)}$ son
facetas para sus subproblemas de empaquetamiento de conjuntos. Para ello
necesitamos considerar un objeto, llamado grafo intersección $G$, asociado al
problema de empaquetamiento de conjuntos. Un resultado debido a Padberg
\cite{pa:facial-structure} nos permite relacionar facetas de SPP con cliqués de
$G$. Este cambio de lenguaje de problema de empaquetamiento a teoría de grafos
nos permite concluir que la mayor parte de las restricciones del SPP son facetas
del mismo, lo que implica que las formulaciones $\rm{(SLL}_1\rm{)}$ y $\rm{(SLL}_2\rm{)}$ no
admiten una mejora trivial.

\paragraph*{}

Finalmente, el quinto capítulo presenta las contribuciones originales del autor.
Éstas se resumen en dos: extensión de las formulaciones $\rm{(SLL}_1\rm{)}$ y
$\rm{(SLL}_2\rm{)}$ al problema CRPP sin envidia y estudio computacional de
dichas extensiones.

El Problema de Asignación de Precios con Rango y Capacidades (en inglés,
\emph{Capacitated Rank Pricing Problem}, CRPP) es una extensión del RPP. Las
adiciones introducidas por CRPP son el permitir definir limitaciones de
inventario y el permitir a cada cliente tener un presupuesto diferente para cada
producto. Las limitaciones de inventario permiten modelar situaciones en que no
se puede producir un bien de manera ilimitada. Por ejemplo, un teatro pude
ofertar 50 localidades en palco y 200 en la platea, pero ninguna más, por
limitaciones de espacio. El CRPP tiene dos variantes: con y sin envidia. Se dice
que una solución presenta envidia si algún cliente $k$ desea adquirir un
producto $i$ cuyo precio es menor o igual que su presupuesto, pero no puede
hacerlo porque se han agotado sus existencias. El CRPP con envidia es aquel en
que se permiten soluciones con envidia. A su vez, CRPP sin envidia se refiere a
aquel en que no se permiten soluciones que presenten envidia. Nosotros
presentamos dos formulaciones para el CRPP sin envidia basadas en las
formulaciones $\rm{(SLL}_1\rm{)}$ y $\rm{(SLL}_2\rm{)}$ antes aludidas.

Por último, el quinto capítulo presenta también un estudio computacional en que
se estudia la eficiencia de las formulaciones propuestas para el CRPP. Las
instancias empleadas para este estudio han sido generadas aleatoriamente.

\selectlanguage{english}

\chapter{Formulations for the Rank Pricing Problem}
\label{chp:problem-formulations}

As said in the introduction, the resolution of linear programming problems
requires developing good formulations. Such a requirement is this chapter's main
motivation, driving the description of its formulations, each one being a small
improvement over the previous one. We will build on the introduction to the RPP
given in Section \ref{sec:int:rpp}, and follow sections 2 and 3 of
\cite{ca:rpp}, from which the formulations of this chapter originate.  The first
formulation, \bnlp, is a conceptually natural treatment of the problem,
explicitly representing the bilevel structure of the RPP as an interaction of a
seller pricing the items and several customers buying them.  From \bnlp we
obtain by introducing a new set of decision variables the formulation \bnlv.

The introduction of both bilevel formulations will aid the reader to get
acquainted with the RPP. Furthermore, the second of these formulations, \bnlv,
will also be the starting point for obtaining the single level formulations \bnl
and \slnl.

\section{Bilevel formulation} % -----------------------------------------------
\label{sec:pf:bilevel-formulation}

\subsection{Formulation with variables $p_i$} % ...............................
\label{ssc:pf:bf:bnlp}

Recall that in the RPP a seller produces some finite set of products, $I$, and
wants to price them so as to maximise the gains he will obtain through their
sale. These products are in turn bought by some customers, which are represented
as elements of a finite set $K$, trying to acquire the most preferred items
lying within their budget. It is thus conceptually natural to model the RPP as a
bilevel problem, so that each level represents one of the mentioned types of
actors, seller and client. On the upper level, the seller prices the items,
maximising his gains. On the lower level, each client tries to acquire his most
preferred item, always respecting his budget. Note that these two problems are
interrelated: by pricing the products, the seller determines which items fall
within each client's budget. Likewise, by purchasing one item or another, the
clients determine the seller's profits.

Turning now to our first formulation, $\ii$ will represent an arbitrary item and
$\kk$ an arbitrary product. Variable $p_i$ is then a real positive value
representing $i$'s price. These $p_i$ will be the decision variables of the
upper-level problem, for their value is decided by the seller. We furthermore
need to consider binary variables $x_i^k$, determining that customer $k$
acquires item $i$ iff $x_i^k = 1$. As said in the introduction, $s_i^k$
indicates the preference level of client $k$ for product $i$, being $s_i^k = 0$
iff $k$ does not want to buy $i$ in any case. Lastly, $b^{\sigma(k)}$ will take
the value of $k$'s budget. With this notation, formulation \bnlp then the form:

{
    \newcommand{\sums} {
        \sum_\kk \sum_\is
    }

    \begin{eqnarray}
        \rm{(BNL}^p\rm{)}
        &\max_p      & \quad \sums p_i x_i^k              \label{eq:bnlp:a}\\
        &\text{s.t.} & \quad p_i \geq 0, \quad \forall\ii \label{eq:bnlp:b}
    \end{eqnarray}
}

\noindent
where $\forall \kk$, $x^k = (x_i^k)_\ii$ is an optimal solution of the
problem:

{
    \newcommand{\sums} {
        \sum_\is
    }

    \begin{eqnarray}
        &\max_{x^k}  & \quad \sums s_i^k x_i^k                  \label{eq:bnlp:c}\\
        &\text{s.t.} & \quad \sums x_i^k     \leq 1             \label{eq:bnlp:d}\\
        &            & \quad \sums p_i x_i^k \leq b^{\sigma(k)} \label{eq:bnlp:e}\\
        &            & \quad x_i^k \in \binset, \forall \is     \label{eq:bnlp:f}
    \end{eqnarray}
}

In this formulation, \eqref{eq:bnlp:a} is the sum of prices of each product
acquired, and thus represents the seller's total revenue. \eqref{eq:bnlp:b}
state that prices are nonnegative real variables. In the lower level problem,
\eqref{eq:bnlp:f} state that the $x_i^k$ are binary variables, and
\eqref{eq:bnlp:d} ensures that client $k$ purchases at most one item. The left
hand side of \eqref{eq:bnlp:e} then takes the value of the price of the item $i$
acquired by $x$, so that the inequality restricts the bought items to be among
those with price falling within $k$'s budget.  The meaning of \eqref{eq:bnlp:c}
becomes then clear: it is the preference level of the item bought by $k$. Thus,
maximising this sum, we assure that $x_i^k = 1$ just for the $i$ most preferred
by $k$ and falling within his budget.

However, in order to prove that this formulation is well-posed, we need to
assure that the lower level problems have a unique solution. Suppose, for
instance, that for some customer $k$ the maximum of the lower level problem were
attained at two different vectors, $x^k$ and $\tilde{x}^k$. This corresponds to
$k$ being able to select two most-preferred products for a given assignment of
prices $p$. This would be problematic, for the employment of $x^k$ and
$\tilde{x}^k$ would give rise to different values for the upper level objective
function \eqref{eq:bnlp:a}. Next result assures that this situation cannot
arise.

\begin{proposition}
    The lower level problems of formulation \bnlp have a unique optimal
    solution.
\end{proposition}

\begin{proof}
    The proof consists on explicitly giving the optimum value of the $x^k$
    vector, and noting that this value is unique.  Let us consider for this
    matter the lower level problem associated to some customer $\kk$ and price
    assignment $p$. If $p_i > b^{\sigma(k)}$ for all products $\ii$, then no
    $x_i^k$ could take a value of 1, for then \eqref{eq:bnlp:e} would not hold.
    Thus, $x_i^k=0, \forall\is$ is the only solution in this case.  Otherwise,
    constraint \eqref{eq:bnlp:d} asserts that at most one $x$-variable $x_i^k$
    may be non-zero, and the optimal solution is then given by $x_j^k=0$ for all
    $j\in S^k$ except for the unique $i$ such that $s_i^k=\max\{s_j^k~|~j\in
    S^k, p_j\leq b^{\sigma(k)}\}$. Note that the uniqueness of solution directly
    depends on the product preferences $s_i^k$ being all distinct.
\end{proof}

\subsection{Formulation with variables $v_i^\ell$} % ............................
\label{ssc:pf:bf:bnlv}

Formulation \bnlp contains both integer (the $x_i^k$) and real (the $p_i$)
variables. However, the possible values for the $p_i$ may be restricted, so that
only a finite set of prices must be considered, thus converting the problem to a
purely integer one. The key insight for this was noted by the authors of
\cite{ru:nonparametric}, and we phrase it here as a lemma:

\begin{lemma}
    There exists solutions of \bnlp with $p_i \in B, \forall \ii$.
\end{lemma}

\begin{proof}
    This proof is adapted from that of \cite{do:envy}. Suppose that there were
    an optimal $p$ and a product $\ii$ such that $p_i$ were strictly between two
    prices in $B$. That is, $b^m < p_i < b^{m+1}$ for some $1 \leq m \leq M -
    1$. If in this solution no client buys $i$, then we may change its price as
    we please, and thus obtain a valid optimal solution where $p_i=b^{m+1}$.

    We will prove by contradiction that the reverse case (some client buys $i$)
    cannot arise. If such a client existed, we could increase $p_i$ up to
    $b^{m+1}$, and the resulting price assignment would leave invariant the sets
    of products that customers can afford. Indeed, all clients $\kk$ with
    $\sigma(k) \leq m$ could not afford $i$ before the change, and cannot afford
    $i$ after it. Likewise, all $\kk$ with $\sigma(k) \geq m + 1$ could afford
    $i$ before the change, and after it they still do. Therefore, the increase
    in $p_i$ is possible. However, this would imply obtaining a feasible
    solution with an optimal value strictly greater than that of the original
    optimal solution, which is absurd.
\end{proof}

Therefore, the set of product prices may be chosen to equal the set of
client budgets. We will exploit this remark by defining a new set of binary
variables, $v_i^\ell, \forall\ii, \forall\lm$, which will represent the price
assigned to the $i$'th item, thus replacing the continuous $p_i$ variables.
Namely, $v_i^\ell = 1$ iff product $i$ is assigned price $b^\ell$. Using these
variables, we can replace \eqref{eq:bnlp:b} with

\begin{eqnarray*}
    \sum_{\ell=1}^M v_i^\ell \leq 1, && \forall \ii,\\
    v_i^\ell \in \binset,            && \forall \ii, \lm
\end{eqnarray*}

\noindent
The first constraints assert, in the same manner as \eqref{eq:bnlp:d}, that each
product can be assigned at most one price. The second constraints assert that
the $v_i^\ell$ are binary variables. We also need to replace constraint
\eqref{eq:bnlp:e} of the lower level problem, restricting the price of the
products affordable by $k$, with the following constraint:

\[
    x_i^k \leq \sum_{\ell=1}^{\sigma(k)} v_i^\ell, \quad \forall \kk, \is
\]

This constraint asserts that $x_i^k$ may only take a value of 1 if $v_i^\ell =
1$ for some $\ell$ such that $1 \leq \ell \leq \sigma(k)$. Since $b^{\sigma(k)}$
is $k$'s budget, and $\ell_1 < \ell_2$ implies $b^{\ell_1} < b^{\ell_2}$, this
means $k$ can only buy $i$ if item $i$ gets assigned a price $b^\ell$ less than
or equal to its budget.

Through these substitutions, we arrive at a purely integer programming
formulation, which we will denote by \bnlv:

{
    \newcommand{\sumk} { \sum_\kk }
    \newcommand{\sumi} { \sum_\is }
    \newcommand{\suml} { \sum_{\ell=1}^{\sigma(k)} }
    \newcommand{\summ} { \sum_{\ell=1}^{M} }

    \begin{eqnarray}
        \rm{(BNL}^v\rm{)}
        &\max_v      & \sumk\sumi(\suml b^\ell v_i^\ell) x_i^k     \label{eq:bnlv:a}\\
        &\text{s.t.} & \summ v_i^\ell \leq 1, \quad \forall\ii     \label{eq:bnlv:b}\\
        &            & v_i^\ell \in \binset, \quad \forall\is, \lm \label{eq:bnlv:c}
    \end{eqnarray}
}

\noindent
with $x^k$ being solutions to the following problem:

{
    \newcommand{\sumi} { \sum_\is }
    \newcommand{\suml} { \sum_{\ell=1}^{\sigma(k)} }
    
    \begin{eqnarray}
        &\max_{x^k}  & \sumi s_i^k x_i^k                                \label{eq:bnlv:d}\\
        &\text{s.t.} & \sumi x_i^k \leq 1                               \label{eq:bnlv:e}\\
        &            & x_i^k \leq \suml v_i^\ell, \quad \forall\kk, \is \label{eq:bnlv:f}\\
        &            & x_i^k \in \binset, \quad \forall\is.             \label{eq:bnlv:g}
    \end{eqnarray}
}

\noindent
Other than the changes already mentioned and leading to constraints
\eqref{eq:bnlv:b}, \eqref{eq:bnlv:c} and \eqref{eq:bnlv:f}, \bnlv\newline distinguishes
itself from \bnlp by its upper-level cost function \eqref{eq:bnlv:a}. This is
obtained from \eqref{eq:bnlp:a} by formally substituting $p_i$ by the term
$
    \sum_{\ell=1}^{\sigma(k)} b^\ell v_i^\ell
$. For each $k$, this sum equals the price given to the $i$'th item, if $k$ can
afford it, and 0 otherwise.

Changing from $p$ to $v$ variables leads to an interesting consequence. As noted
in the paper \cite{ca:rpp}, the lower level of \bnlv can be incorporated to the
upper level problem as a group of constraints, thus converting \bnlv in a
single-level formulation. The specific way in which this may be archived will be
explained in the next section.

\subsection{Single level formulation} % .................................................
\label{ssc:pf:bf:bnl}

One important remark allowed the authors of \cite{ca:rpp} to eliminate the
lower level problem of \bnlv. This remark was the unimodularity of the matrix
associated to these problems. As noted in the introductory chapter, if an
integer formulation has a totally unimodular associated matrix, then the
solution of its linear relaxation has integer coefficients, thus being possible
to solve the integer problem as a linear one. The author has developed the
following proof, which assures that this is indeed the case of our lower level
problems:

\begin{lemma}
    The lower level problems of \bnlv have totally unimodular associated
    matrices.
\end{lemma}

\begin{proof}
    We note that the constraint matrix $A$ takes in this case the form

    \begin{equation*}
        \begin{pmatrix}
            1      & 1      & 1      & \cdots & 1     \\
            1      & 0      & 0      & \cdots & 0     \\
            0      & 1      & 0      & \cdots & 0     \\
            0      & 0      & 1      & \cdots & 0     \\
            \vdots & \vdots & \vdots &        & \vdots\\
            0      & 0      & 0      & \cdots & 1     \\
            1      & 0      & 0      & \cdots & 0     \\
            0      & 1      & 0      & \cdots & 0     \\
            0      & 0      & 1      & \cdots & 0     \\
            \vdots & \vdots & \vdots &        & \vdots\\
            0      & 0      & 0      & \cdots & 1     \\
        \end{pmatrix}
    \end{equation*}

    \noindent
    We may then apply result \ref{pro:tu-characterisation} for concluding that
    $A$ is totally unimodular. Indeed, for any set of columns $J$, we may choose
    a partition $J_1$, $J_2$ with cardinalities $|J_1|$ and $|J_2|$ with a
    difference of at most one unit. Then,
    \begin{itemize}
        \item
            For $i = 1$,
            \[
                \left| \sum_{j \in J_1} a_{1j} - \sum_{j \in J_2} a_{1j} \right|
                    = \left| |J_1| - |J_2| \right|
                    \leq 1
            \]
        \item
	    For $i \neq 1$, row $a_{i\cdot}$ contains exactly one non-zero
	    entry, and thus
            \[
                \left| \sum_{j \in J_1} a_{1j} - \sum_{j \in J_2} a_{1j} \right|
                    = 1
            \]
    \end{itemize}
\end{proof}

This result assures us that constraints \eqref{eq:bnlv:f} may be relaxed to $0
\leq x_i^k \leq 1, \forall\is$. Furthermore, once the prices (i.e. the
$v_i^\ell$ variables) are fixed, the set of products affordable by each customer
$k$ can be explicitly calculated without resorting to linear constraints as
$
    I(k) = \{ i \in S^k~|~\sum_{l=1}^{\sigma(k)} v_i^l = 1 \}
$.
These two remarks allow us to write the lower level problem in the following
manner:

\begin{eqnarray*}
    &\max_{x^k} & \sum_{i \in I(k)} s_i^k x_i^k\\
    &\text{s.t.}& \sum_{i \in I(k)}       x_i^k \leq 1\\
    &           &                         x_i^k \geq 0, \quad i \in I(k).
\end{eqnarray*}

Note that it is not necessary to explicitly add the restriction $x_i^k \leq 1$,
for this is already implied by
$
    \sum_{i \in I(k)} x_i^k \leq 1
$.
Now, we may follow the concepts from duality theory presented in Subsection
\ref{ssc:int:lp:duality-theory} in order to give for each $\kk$ the dual problem
of the lower-level problem:

\begin{eqnarray*}
    &\min_{u^k}  & \quad u^k\\
    &\text{s.t.} & \quad u^k \geq s_i^k, \quad i \in I(k)\\
    &            & \quad u^k \geq 0.
\end{eqnarray*}

\noindent
Duality theory now allows us to characterise the primal and dual problems'
solutions $x^k$ and $u^k$ through the following conditions:

\begin{eqnarray}
    \sum_{i\in I(k)} s_i^kx_i^k &=   & u^k                            \label{eq:c1}\\
                          u  ^k &\geq& s_i^k \quad \forall i \in I(k) \label{eq:c2}\\
    \sum_{i\in I(k)}      x_i^k &\leq& 1\notag\\
                   x_i^k, u  ^k &\geq& 0\notag.
\end{eqnarray}

Note that equation \eqref{eq:c1} and inequalities \eqref{eq:c2} may be combined
into a single inequality not containing $u^k$, so that the system of
inequalities results in a set of conditions over $x^k$ alone. These relations
hold the key to removing the lower-level problem from \bnlv and for obtaining the
following formulation, for their inclusion in the upper level problem will
result in $x_k$ having the same meaning as that given by being solution to the
lower level problem. We thus arrive to the following single-level formulation
for the RPP:

{
    \newcommand{\sumk} { \sum_\kk }
    \newcommand{\sumi} { \sum_\is }
    \newcommand{\sumj} { \sum_{j\in S^k} }
    \newcommand{\suml} { \sum_{\ell=1}^{\sigma(k)} }
    \newcommand{\summ} { \sum_{\ell=1}^{M} }

    \begin{eqnarray}
        \rm{(BNL)}
        &\max_{v,x}  & \sumk\sumi(\suml b^\ell v_i^\ell) x_i^k                            \label{eq:bnl:a}\\
        &\text{s.t.} & \summ v_i^\ell \leq 1, \quad \forall\ii                            \label{eq:bnl:b}\\
        &            & \sumi x_i^k    \leq 1, \quad \forall\kk                            \label{eq:bnl:c}\\
        &            & x_i^k \leq \suml v_i^\ell, \quad \forall\kk, \ii                   \label{eq:bnl:d}\\
        &            & \sumj s_j^k x_j^k \geq s_i^k\suml v_i^\ell, \quad \forall\kk, \is  \label{eq:bnl:e}\\
        &            & v_i^\ell, x_i^k \in \binset, \quad \forall\kk, \is, \lm            \label{eq:bnl:f}
    \end{eqnarray}
}

The problem is now over two sets of variables, $v$ and $x$. Cost function
\eqref{eq:bnl:a} is the same as \eqref{eq:bnlv:a}, and constraints
\eqref{eq:bnl:b} correspond to \eqref{eq:bnlv:b}. Conditions \eqref{eq:bnl:c}
and \eqref{eq:bnl:d} correspond to \eqref{eq:bnlv:e} and \eqref{eq:bnlv:f}
above. Finally, constraints \eqref{eq:bnl:e} have been obtained from the
conditions above mentioned, by combining \eqref{eq:c1} and \eqref{eq:c2} into a
single inequality, so that explicit reference to $u^k$ can be removed.

Thus, if some vectors $v$, $x$ satisfy conditions \eqref{eq:bnl:b} to
\eqref{eq:bnl:f}, $v$ will be a valid price assignment, and all $x^k$ will
verify \eqref{eq:c1} and \eqref{eq:c2}, thus being a solution to the lower level
problem of \bnlp. Therefore, BNL is a valid formulation for the RPP.

\section{Alternative single level formulation} % -------------------------------
\label{sec:pf:alternative}

We will now use the concepts and terminology of Section
\ref{sec:pf:bilevel-formulation} to introduce a new single level formulation for
the \rpp which will serve as the basis for the further development of this work.
Firstly, we will need to define two families of sets which will be used in the
new formulation.

\begin{definition}
    For any customer $\kk$ and products $i,j\in S^k$, we say that $i$ is
    $k$-better than $j$ if $k$ prefers product $i$ over product $j$. We denote
    the set of products $k$-better than $i$ as
    $
        B(k,i) = \{j\in S^k \mid j\text{ is } k\text{-better than } i\}
    $.
    
    Similarly, $i$ is defined to be $k$-worse than $j$ if $k$ prefers product
    $j$ over product $i$. The set of products $k$-worse than $i$ is denoted by
    $
        \overline{B(k,i)} = \{j\in S^k \mid j\text{ is } k\text{-worse than } i\}
    $.
\end{definition}

We may use these sets in the following manner. A customer $k$ buys product $i$
iff $i$'s price is within $k$'s budget, and all products $j$ that are $k$-better
than $i$ are above $k$'s budget. Using the price-indicator variables $v_j^\ell$,
this is equivalent to asserting that $x_i^k=1$ iff

\[
    \sum_{\ell=1}^{\sigma(k)} v_i^\ell = 1 \text{ and }
        \sum_{\ell=1}^{\sigma(k)} v_j^\ell = 0, \forall j\in B(k, i).
\]

This property lends itself towards defining constraints (4d) and (4e) in the
following single level non-linear formulation \slnl:

\begin{eqnarray}
    \rm{(SLNL)}
    &\max_{v,x}  & \sum_\kk\sum_{i\in S^k} (\sum_{\ell=1}^{\sigma(k)} b^\ell v_i^\ell)x_i^k \label{eq:slnl:a}\\
    &\text{s.t.} & \sum_{i\in S^k} x_i^k \leq 1, \quad\forall\kk                            \label{eq:slnl:b}\\
    &            & \sum_{\ell=1}^M v_i^\ell \leq 1, \quad\forall\ii                         \label{eq:slnl:c}\\
    &            & x_i^k + \sum_{\ell=1}^{\sigma(k)} v_j^\ell \leq 1, \quad \forall\kk, i \in S^k, j\in B(k, i) \label{eq:slnl:d}\\
    &            & x_i^k + \sum_{\ell=\sigma(k)+1}^M v_i^\ell \leq 1, \quad \forall\kk, i \in S^k               \label{eq:slnl:e}\\ 
    &            & v_i^\ell, x_i^k\in\{0,1\}, \forall\kk, i\in S^k, \ell\in\{1, \ldots, M\}.                    \label{eq:slnl:f}
\end{eqnarray}

The objective function \eqref{eq:slnl:a} is the same as that of \bnl,
\eqref{eq:bnl:a}. Restriction families \eqref{eq:slnl:b} and \eqref{eq:slnl:c}
correspond, respectively, to \eqref{eq:bnl:c} and \eqref{eq:bnl:b}. Restriction
family \eqref{eq:slnl:e} results of combining \eqref{eq:bnl:d} and
\eqref{eq:bnl:b}, and for a customer $k$ and a product $i$, it means that if $k$
buys $i$, then $i$'s price is not higher than $k$'s budget. Finally, constraint
family \eqref{eq:slnl:d} is obtained from the remark above, and means that if
$k$ buys $i$, then it does not buy any product $j$ that is $k$-better than $i$.

Finally, the authors of \cite{ca:rpp} noted that constraints \eqref{eq:slnl:d}
can be strenghened by considering the following family of constraints:

\begin{proposition}
    The following inequalities are valid constraints for the problem \slnl and
    dominate constraints \eqref{eq:slnl:d}:

    \begin{equation}
        \sum_{j\in\overline{B(k, i)}} x_j^k
            + \sum_{\ell=1}^{\sigma(k)} v_i^\ell \leq 1
            , \quad \forall \kk,
                            i \in S^k \text{ such that }
                                \overline{B(k, i)} \neq \varnothing.
        \label{eq:slnl:valid-inequalities}
    \end{equation}
\end{proposition}

\begin{proof}
    First, we will prove that constraints \eqref{eq:slnl:valid-inequalities} are
    valid. i.e. that every solution to \slnl verifies them. If product $i$ is
    within customer $k$'s budget, then $\sum_{\ell=1}^{\sigma(k)} v_i^\ell=1$.
    Furthermore, none product $k$-worse than $i$ will be bought, and so
    $\sum_{j\in \overline{B(k, i)}} x_j^k=0$. Thus,
    \eqref{eq:slnl:valid-inequalities} holds in this case. If product $i$ is not
    within $k$'s budget, then $\sum_{\ell=1}^{\sigma(k)} v_i^\ell=0$.
    Furthermore, at most one product $k$-worse than $i$ will be bought, and so
    $\sum_{j\in \overline{B(k, i)}} x_j^k \leq 1$. Thus,
    \eqref{eq:slnl:valid-inequalities} always hold.

    Secondly, we will prove that \eqref{eq:slnl:valid-inequalities} dominate
    \eqref{eq:slnl:d}. For that, let $\kk$, $\is$ and $j \in B(k, i)$, so that
    $x_i^k + \sum_{\ell=1}^{\sigma(k)}v_j^\ell \leq 1$. Since $j$ is $k$-better
    than $i$, $i$ is $k$-worse than $j$, and thus $x_i^k \leq \sum_{i' \in
    \overline{B(k,j)}} x_{i'}^k$. Therefore,

    \[
        \sum_{j\in\overline{B(k, i)}} x_j^k
            + \sum_{\ell=1}^{\sigma(k)} v_i^\ell \leq 1.
    \]
\end{proof}
